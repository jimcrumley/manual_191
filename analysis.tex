% DATA ANALYSIS
\newexp

\section*{Before Lab}
You must make three careful graphs in your lab notebook
{\bf before} you come to lab: 
\begin{itemize}
\item Data Set 1 --- the graph of distance (on the $y$-asis) vs.\ time (on the
$x$ asis) should look approximately linear.
\item Data Set 2 --- the graph of $N$ (on the $y$-asis) vs.\ time (on the
$x$ asis) should look curved.
\item Data Set 2 --- the semi-log graph of $\ln(N)$ (on the $y$-asis) vs.\ time (on the
$x$ asis) should look linear.  Follow the exponential function graphical
analysis outlined on page \pageref{exprel} in Appendix C. 
Provide a table analogous to Table \ref{table:pressure} on page
\pageref{table:pressure} showing the logarithmic transformation of the data and error
in Data Set 2.
\end{itemize}
Each graph should be sized to fill a notebook
page with axes ranges selected to surround the data with a minimum of unused space.
Accurately drawn errors bars (following Figure~ref{fig:b1}) are required.


One purpose of the first few experiments is to introduce you to
the data reduction and analysis concepts that are described in the
four Appendices to this manual.  You should begin working your way
through these Appendices now---there is a lot in them and it will take
you some time to understand it.

For this experiment you will need to be familiar with at least some
of the material in each of these Appendices:
\begin{itemize}
\item Appendix A (on error analysis) talks about how to estimate uncertainties
in experimentally measured quantities.
\item Appendix B (fairly short) gives you some guidelines on how to make a graph.
Use the graphs in Appendix C as examples.
%
\item Appendix C (on graphical analysis), especially the sections on linear
and exponential functions.  Note especially the logarithmic transformation that we
use to make an exponential function look like a straight line.  You may need
to review the properties of logarithms as you study this section.
%
\item Appendix D on Computer-Assisted Curve Fitting.  We will be making
two least-squares fits on the computer for this experiment, and a good many
more later in the semester.  Appendix D provides an introduction to
the method of least squares and to the computer program we will be
using.
\end{itemize}

Of course, you should also read carefully the writeup for this experiment, given
below.  

%\begin{tabbing}
%reference12345 \= col2 \kill
%References: \> Laboratory Manual, Experimental Error or Uncertainty,
% page~\pageref{scierror} \\
%\> Data Handling, page~\pageref{datahandle} \\
%\> Data Analysis, page~\pageref{datanal} \\
%\> Computer Assisted Curve Fitting, page~\pageref{compassis}
%\end{tabbing}

\section*{Data}
    Much of your work in laboratory this semester will involve
the careful plotting and analysis of experimental data.  In this
exercise you will be given two sets of data and asked to
work with each set.
%\begin{center}
%Data Set 1 \\
%Average Velocity
%\end{center}
\subsubsection*{Data Set 1:  Average Velocity}
     Suppose an airplane on a long flight over a very poorly
mapped route passes over a series of air traffic control radio
beacon checkpoints that have been dropped by parachute.  At each
checkpoint the time is recorded with an accuracy of about 1
second (i.e., better than 0.001 hours).  The distance of each
checkpoint from the plane's starting point is known only to
within about $\pm$150 km.  The following data are recorded:
\newcommand{\Z}{\phantom{0}}
\begin{center}
\begin{tabular}{|cc|}\hline
 & distance from \\
 elapsed time & starting point \\
 (hours $\pm$0.001) & (km $\pm$150) \\ \hline
 0.235 & \Z200 \\
 1.596 & 1200 \\
%1.678 & 1500 \\
 1.921 & 1850 \\
 2.778 & 2100 \\
 3.310 & 2750 \\
 4.314 & 3500 \\
 4.846 & 4200 \\
 6.147 & 4900 \\
 6.678 & 5700 \\
 7.979 & 6650 \\\hline
\end{tabular}
\end{center}
%\begin{center}
%Data Set 2 \\
%Half-life
%\end{center}
\subsubsection*{Data Set 2:  Half-life of a radioactive element}
    The activity of a certain radioactive element is monitored by
a Geiger counter, which clicks and records a ``count" each time it
detects a nucleus undergoing a radioactive decay.  The number $N$ of counts
during a one minute period is recorded along with the start time of
the count.  The estimated uncertainty in a count is
$\sqrt{N}$. (Why this is the proper error estimate is explained
in Taylor.)  The error in start-time is negligible.
%<<Statistics`DiscreteDistributions`
%Table[Random[PoissonDistribution[2800 E^(- .03 n)]],{n,1,50}]
\begin{center}
\begin{tabular}{|ccc|}
\hline
time & $N$ & $\delta N$ \\
(min) & (counts/min) & (counts/min) \\ \hline
\Z2 & 2523 & 50\\
\Z4 & 1908 & 44\\
\Z6 & 1632 & 40\\
\Z8 & 1263 & 36\\
10  & 1068 & 33\\
12  & \Z861 & 29\\
14  & \Z714 & 27\\
16  & \Z560 & 24\\
18  & \Z452 & 21\\
20  & \Z372 & 19\\ \hline
\end{tabular}
\end{center}
%                FIT                 
% PARAMETER     VALUE      ERROR     
%   A =         0.303E+04   50.      
%   B =        -0.105      0.18E-02  
%\Z2 & 2662 & 52 \\
%\Z6 & 2333 & 48 \\
%10 & 2091 & 46 \\
%14 & 1817 & 43 \\
%18 & 1719 & 41 \\
%22 & 1464 & 38 \\
%26 & 1328 & 36 \\
%30 & 1087 & 33 \\
%34 & 1034 & 32 \\
%38 & \Z886 & 30 \\
%42 & \Z733 & 27 \\
%46 & \Z698 & 26 \\
%50 & \Z649 & 25 \\\hline
%\end{tabular}
%\end{center}
%%                FIT                
%% PARAMETER     VALUE      ERROR     
%%   A =         0.283E+04   41.     
%%   B =        -0.303E-01  0.61E-03 
%% reduced chi-squared of  1.4

%The activity $C$ in
%counts/minute are recorded as a function of time, resulting in
%the following table.  
%The uncertainty in the activity is about
%3\%.
%\begin{center}
%\begin{tabular}{cc|cc}
%time & $C$ & time & $C$ \\
%(min) & (counts/min $\pm$3\%) & (min) & (counts/min $\pm$3\%) \\ \hline
%5.5 & 2410 & 26.5 & 1251 \\
%7.0 & 2284 & 29.5 & 1137 \\
%8.5 & 2188 & 32.5 & 1130 \\
%10.0 & 2038 & 35.5 & 1002 \\
%11.5 & 1959 & 38.5 & 905 \\
%13.0 & 1971 & 41.5 & 786 \\
%14.5 & 1918 & 44.5 & 773 \\
%16.5 & 1756 & 47.5 & 704 \\
%18.5 & 1675 & 50.5 & 687 \\
%20.5 & 1599 & 53.5 & 618 \\
%22.5 & 1464 & 56.5 & 562 \\
%24.5 & 1350 & 59.5 & 478 \\
%\end{tabular}
%\end{center}
%                FIT                      
% PARAMETER     VALUE      ERROR          
%   A =         0.280E+04   37.            
%   B =        -0.290E-01  0.40E-03        

\section*{Analysis}
Try to complete all four steps given below {\em before} coming to lab.
If you have problems, talk to your instructor, but as a minimum
draw the three required graphs of the above data in your notebook.


\begin{enumerate}
\item In your lab notebook, make a careful graph of each set of
        data, following the guidelines given in
        Appendix B  and using the graphs in
        Appendix C  in the lab
        manual as an example.  Be sure
        you plot the error bars.
\item For the first data set, the graph of distance vs.\ time
        should be linear.  If it is, use the graph to estimate
        the average speed of the plane.

        Make careful calculations of the slope and the $y$-intercept
        of graph, and explain how they are to be interpreted.
        Use the range of
        possible slopes implied by the error bars to estimate the
        uncertainty in average speed (see Fig.~\ref{fig:slope} in
Appendix C) and $y$-intercept. %page~\pageref{datanal}).
\item For the second data set, the graph of activity vs.\ time
        should {\em not} be linear.  Therefore the first step is to
        determine what sort of function describes the data.  Read
        carefully the section in Appendix C on Exponential Function
        (page~\pageref{exprel}) of the lab manual.  Following
        the advice of that section make a semi-log plot of the
        data which should look linear.  Begin by making a table analogous to 
Table \ref{table:pressure} on page
\pageref{table:pressure} showing the logarithmic transformation of the data and error
bars.
Calculate the $A$ and $B$
        parameters and their uncertainties.


%        and see if you can find a way of plotting the data that
%        gives you a straight line.  (Hint:  Radioactive materials
%        usually decay exponentially).

%        Note that you can use Linfit, the data analysis and least-squares
%        fitting program we will be using, to make on-screen graphs of
%your
%data quickly and easily.  It may save you a good bit of time if you
%        use the program to investigate the form
%        of the fitting function before plotting the graphs in your
%        lab notebook.
%
\item   Exponential functions can be characterized by a ``half-life,'' $\tau$,
        in this case the time
        needed for half of the original sample to undergo decay.
The half-life can be easily estimated from your graphs.  Find
any two points on the smooth curve (i.e., not data points) such that the activity 
of the second point
is half the activity of the first point.  The time interval between
the two points is the half-life.  Use your graph to make a rough estimate.

We can find a more accurate estimate of the half-life by deriving a 
relationship between it and the $B$ parameter.
We have two points $(t_1,N_1)$ and $(t_2,N_2)$ such that
$N_2=N_1/2$ and $\tau=t_2-t_1$.  The exponential relation says:
\begin{eqnarray*}
N_1&=&A e^{Bt_1}\\
N_2&=&Ae^{Bt_2}
\end{eqnarray*}
If we divide the second equation by the first we have:
\[{1\over 2} =  {N_2\over N_1} = {Ae^{Bt_2} \over Ae^{Bt_1}} = e^{B(t_2-t_1)} =
e^{B\tau} \]
If we now take the natural log of both sides we have:
\[\ln\left({1\over 2}\right) = \ln\left(e^{B\tau}\right) = B\tau \]
or
\[B = {\ln\left({1\over 2}\right) \over \tau} \]
or, solving for $\tau$,
\[\tau = {\ln\left({1\over 2}\right) \over B} \].

Thus if you calculate $B$ using your graph, you can use the result 
to calculate the half-life $\tau$.
Note that this method of calculating $B$ is actually the same that
described in Eq.~\ref{eq:cnew}:
\[
B = 
{\ln\left(y_{2}/y_{1}\right) \over x_{2} - x_{1}} =
{Y_{2}-Y_{1} \over x_{2} - x_{1}}
\]
Go on to calculate the half-life $\tau$, and compare it to the value
you found above, from the direct inspection of your graph.

\end{enumerate}
    In the laboratory we will go over the graphical data analysis
outlined above, and then see how to use the computer to analyze
the same data using the method of least squares (see Appendix D).

    Before leaving lab, make sure the instructor has checked your
work.

\section*{Lab Writeup}

The Introduction to this lab manual describes what should
usually be included in your notebook, but this experiment is
a bit different.

For this particular experiment, be sure to include the following items:
\begin{enumerate}
\item The four steps outlined above, which constitute your graphical
analysis of the two data sets.
\item The results of your computer analyses for the two data sets.
Tape the Fit reports and graphs into your notebook.
      Note that Linfit gives values for the {\it uncertainties} in
      the fit parameters (for example, the slope and intercept).
      These uncertainties are important parts of your report!  

Compare
       your own graphical analysis calculation of the
$A$ and $B$ parameters to the one Linfit finds.  The two
       values should agree within experimental uncertainty.

\item Using the $B$ parameter found by Linfit calculate the half-life and its
uncertainty. The calculation of the uncertainty in the
       half-life is probably new to you.  You will need to use one
       of the formulas in Table~\ref{table1} in Appendix A
       (page~\pageref{errorprop}) of the lab manual to find the
       {\em uncertainty} in the half-life in terms of the
       uncertainty in the parameter $B$.  
Check with your instructor if you are not
       sure how this calculation works.  Compare this calculated
half-life to that found graphically.

\item Conclusions (see below)
\item Critique (see below)
%\item $3\times5$ quick report card
\end{enumerate}

\section*{Conclusions}
Your conclusion should include the following:
\begin{itemize}
\item Two tables summarizing the results of your analysis for the
two sets of data (including every calculated $A$ and $B$ value
with units and uncertainty).  
(Tabulating and comparing results obtained during lab is almost
always part of your conclusion section.)
%
\item Several paragraphs discussing the conclusions you drew from
this analysis.  For example,
\begin{itemize}
\item Did your graphical results and computer least-squares fit results agree, within
      experimental uncertainty?
%
\item How valid and reliable were your least squares fits?  See Appendix D
for a discussion of the relationship between reduced
chi-squared and the validity of the fit.
%
\item Any other conclusions you drew from your analysis.
\end{itemize}
\end{itemize}


\subsection*{Critique of Lab}
     Follow the suggestions given in the Introduction to the Laboratory Manual.
You may also want to comment on the relative ease of using the
computer.
