%APPENDIX D--COMPUTER ASSISTED CURVE FITTING
\newapp
%\section*{Computer Assisted Curve Fitting}
%\addcontentsline{toc}{subsection}{Computer Assisted Curve Fitting}
\label{compassis}
\vspace{-.5in}
\section*{Introduction}

The preceding appendix showed how to use graphical techniques to find
out whether a set of data is consistent with a power law or
exponential function.  It also showed how to find the parameters $A$
and $B$ (and their units!) that describe those functions.

Although these methods are powerful, they nonetheless have serious
limitations.  They are helpful in gaining
a good understanding of what is involved in fitting a set of data to a
function.  But often our data will not be consistent with any of the
functions we have talked about so far---straight line, power law, or
exponential.  (A simple example is the quadratic function, $y = A
+ B x + C x^2$.)  Moreover, we might hope to find more systematic ways
of
finding the {\em uncertainties} in the parameters that characterize a
given function than we were able to develop in Appendix C.

A solution was published in 1805 by Adrien Marie Legendre in his book 
on determining the orbits of comets. He correctly noted that there is 
arbitrariness in the choice of the best equation, but proposed a solution:
%

\begin{quote}
Of all the principles that can be proposed for this purpose, I think there 
is none more general, more exact, or easier to apply, than that which we have 
used in this work; it consists of making the sum of the squares of the errors 
[deviation of measurement from equation] a minimum. By this method, a kind of 
equilibrium is established among the errors which, since it prevents the extremes 
from dominating, is appropriate for revealing the state of the system which most 
nearly approaches the truth.
\parskip=0pt
\end{quote}

This appendix will introduce you to this technique.
It is variously called the ``method of least squares,'' or
sometimes, ``regression analysis.''  (Social scientists are likely to
use the second phrase.)  In this appendix, we will give a brief
description of the method of least squares.  We will also introduce
you to a computer program, Linfit, that we will be using to do
least-squares fits.

\section*{References}

Our treatment of the method of least squares in this appendix is
necessarily brief and incomplete.  At some point in your career, you
are likely to need to learn more
than can be provided here about least-squares fits.  
A good introduction is the book by
Taylor that we have already mentioned several times in this manual:

\begin{itemize}
   \item     John R.  Taylor, {\em An Introduction
to Error Analysis}, second edition, (University Science Books, 1997).
\end{itemize}

The Physics Department requires this book for all second year and
higher laboratory courses, and we recommend it for anyone planning to
major in physics or engineering.  Copies are available in the libraries,
and should also be available or easily ordered at both bookstores.

Two other useful if more advanced books are:

\begin{itemize}
\item Philip R.  Bevington \& D. Keith Robinson, {\em Data Reduction and
Error Analysis for the
Physical Sciences} (McGraw-Hill, 1992); and
%
\item  William H. Press et al,
{\em Numerical Recipes in Fortran}, second edition, (Cambridge
University Press, 1992) especially chapter 15.
  Much of the Fortran
code used in Linfit is drawn from the first two sources.
\end{itemize}
In
addition, many statistics texts and texts on experimental design and
data analysis have sections on least-squares fitting.


\section*{Method of Least Squares}

One often wants to find the best equation that describes a set of
experimental data.  The method of least squares 
is the most commonly used technique for 
numerical fitting data to a function.  Suppose, for example, that we
have a set of experimental data that lie roughly along a straight
line.  We want to find the ``best" values of slope $B$ and intercept
$A$
for the equation
\[
y = A + Bx
\]

 that will describe the data.  The problem is that the
data don't usually fall exactly on any line.  Moreover,
the data points generally have experimental uncertainties
associated with them.  As a result, any of
several lines (that is, several different sets of values for $A$ and
$B$) are usually plausible.  For straight-line functions, one can do
reasonably well fitting ``by
eyeball,"---that is, using your best judgment in drawing what appears to
be the best line that describes the data.  But it would be nice
to be able to do a little better.  The method of least squares
provides
a commonly used mathematical criterion for finding the ``best" fit.
Suppose, for example, that our data points are 
$(x_i, y_i)$ for $i=1,2,\ldots N$.

For any point, the quantity
\[
y_i -(A + Bx_i)
\]

(often called a "deviation") is the difference between the
experimental value $y_i$ and the value calculated from the equation 
(a
straight line) at the corresponding location $x_i$.  (Note that
this quantity should be approximately the same as your experimental
uncertainty or ``error bar.") The method of least squares finds the
``best" fit by finding the values of $A$ and $B$ that minimize these
deviations not just for a single data point but for all the
data.  More specifically, it does so by finding the minimum in the
quantity that mathematicians call the reduced chi-squared statistic:


\begin{equation}
\bar{\chi}^{2} \approx {{1} \over {N}} \; \sum_{i} {{\left[y_{i} -
         f(x_{i})\right]^{2}} \over {\sigma_{i}^{2}}} .
         \label{eq:chisq}
\end{equation}

For our special case in which the function is a straight line, this
equation becomes

\[
\bar{\chi}^{2}= {{1} \over {N-2}} \; \sum_{i} {{\left[y_{i} -
         (A + Bx_{i})\right]^{2}} \over {\sigma_{i}^{2}}}
\]

where $N$ is the number of data points, $f(x)$ is the fitting
function,
and the $\sigma$'s are the standard deviations introduced in Appendix
A---more intuitively, the uncertainties (or ``errors") associated
with
each $y$ value.  In the second expression, $N-2$ is the 
``number of degrees of freedom" for $N$ data points fit to a line.
For a brief discussion of how one does the minimization
mathematically,
see the section ``What are ``linear" and ``non-linear" fits?'' in the
on-line help in the fitting program Linfit.  See
the References section above for references to more detailed
discussions.  As a practical matter, the computer program will do the
calculations involved, and report the values of the fitting parameters
$A$, $B$, \ldots that best describe your data.

Once a fit has been found, we need some way of determining
how good the fit is.  If, for example, we inadvertently attempt
to fit data that lie along some curve to a straight line
fitting function, our fitting program will give an answer, even though
the fit is a poor one in the sense that the fitting function
will not go through many of the error bars.
One good visual check is to compare the fitted
solution to the data graphically --- that is, plot the function and
your data on the same graph, and see how well the function
describes your data.  Another easy way to check the fit is to see
if the deviations --- that is, the difference between the data
points $y_{i}$ and the calculated values $A + Bx_{i}$ --- are significantly
larger than the corresponding uncertainty.  These deviations are listed on the 
computer-generated Fit report,
and will be easy to inspect.

{\bf IMPORTANT:}  It turns out that our reduced chi-squared statistic
provides another, more abstract criterion for goodness of fit.
Statistical theory tells us that a good fit corresponds to a reduced
chi-squared value of about one.  A proof is beyond the scope of
this manual; see Taylor or any good statistics text for a more
complete discussion.

Nevertheless, it is not hard to understand this result intuitively.
 Note that the quantity we are minimizing, the reduced
chi-squared value, depends on the size of the experimental
uncertainties (that is, the $\sigma$'s in Eq.~\ref{eq:chisq} above)
as well as the
deviations.   Thus, the reduced chi-squared value will be about one
when the deviations --- the differences between the $y$ values
and the fitting function, represented in Eq.~(\ref{eq:chisq}) by the
quantities $ \left[y_{i} -  (A + Bx_{i})\right] $ --- are about
equal to the size of the $y$ error bars, the $\sigma_i$.

Take a look at this equation, and be sure you understand
this point.  Larger values of the $\sigma$'s in this equation will
lead to a smaller reduced chi-squared statistic, and vice versa.

More or less arbitrarily, the fitting
program Linfit assumes that any value of reduced chi-square
($\bar{\chi}^2$) between 0.25 and 4 represents a
reasonably good fit.  Otherwise, the program warns you that your fit
may not be valid.

There are two reasons why the value of the reduced
chi-squared statistic might not be close to one.  First, you may have
picked an inappropriate fitting function: If your data describe a
curve and you try to fit them to a straight line, for example, you
will certainly get a bad fit!  Look at a graph of your data along with
the fitting function and see how well the two agree.

Second, you may have overestimated or underestimated your experimental
uncertainties.  In this case, your fit may be all right, but your
uncertainties are nevertheless leading to a value of reduced
chi-squared significantly larger or smaller than one.  In this case,
{\bf DO NOT} try to correct the problem by guessing at the errors until you
get a reasonable reduced chi-squared value!!  Instead, see if you can
figure out why your estimates of the experimental error are off.  If
you can't, talk to your instructor or simply say in your lab notebook
that your estimates of experimental error are apparently wrong, and
you are not sure why.  This approach is not only honest, it is much
more in the spirit of scientific inquiry!  And as a practical matter,
whoever grades your lab is more likely to be favorably impressed by a statement
that you don't understand something than by an obvious attempt to
``fudge" your error estimates!

\subsection*{Using Errors in both $x$ and $y$}

Most treatments of the method of least squares, and most computer
programs that use it, assume that the only experimental uncertainties
are in $y$ --- in other words, that any uncertainty in $x$ can be
neglected.  This assumption is reasonably accurate in a surprising
number of cases.  Nevertheless, in many experiments there will be
significant uncertainties in both the $x$ and $y$ values.  The program
Linfit does allow the use of such uncertainties for a few of the more
widely used fitting functions.  In future versions, we hope to expand
this list of functions.



\newpage
\section*{Introduction to Linfit}

There are many computer programs that can do least-squares fits.  In
this laboratory, we will for the most part use the program Linfit.
Linfit is available on all Physics Department computers.  Computers 
that run the Windows XP
operating system should have no difficulty in running Linfit.

People living off-campus can
also install their own copy.  It's easy to do---just download  
the installation program from Professor Gearhart's web 
site (http://faculty.csbsju.edu/cgearhart/).  You can also start at the
Physics Department site (http://www.csbsju.edu/physics/) and work your
way down.  Please see the Physics Department
Laboratory Manager or Professor Gearhart if you have any difficulty
downloading the program or installing it.

Your instructor or lab TA should be able to tell you where to find
Linfit on the Windows Start button.  (The location occasionally
changes, so we have not included it here.)

The rest of this Appendix gives a brief introduction to Linfit.  The
discussion is largely based
on the on-line help available in Linfit.  To use the
on-line help, simply choose the Help menu item or press the F1
function key.  Note that on-line help topics may be printed if you
need a hard copy of any particular help topic.

\section*{Overview of Linfit}

Linfit is centered around a Home screen: Starting from
the Home Screen, users may use on-screen ``buttons" to choose an error
mode, enter data, choose a fitting function, make a graph of the data,
or do the fit.  In some cases, these buttons are inactive and colored
red until they are usable.  For example, one cannot make a graph
before data have been entered!  When these buttons become active,
their color changes from red to green.

The default error mode in Linfit is to use only uncertainties in $y$.
However, you can change this default to allow the use of both $x$ and
$y$ uncertainties by making the appropriate choice on the Error Type
menu on the Home screen.

Linfit also has a spreadsheet that is often useful for calculations
and data reduction.

Pressing a button typically transfers you to a new screen.  Once you
are familiar with the program, it is often convenient to use the Window menu
item found on most screens to transfer from one screen to another,
rather than return to the Home screen each time.

Descriptions of some of these screens are found below.  It is probably
best to read these sections while you are actually using the program.

\subsection*{Error Mode Screens}

On these screens, you are prompted to choose a button corresponding to
the nature of the experimental uncertainties in your experiment.  For
example, the uncertainty can be a percentage of the $x$ or $y$ value,
the same constant error for each point, and so on.  For most entries,
you will be prompted for a numerical value.  Press the Continue button
to return to the Home screen.

\subsection*{Data Entry Screen}

One uses this screen to enter or edit data.  The buttons on this
screen perform the following tasks:


\begin{tabular} {l p{3in}}
\underline{Enter Data}     &   This button prompts the user to enter
                               data either from the keyboard or from a
                               disk file.  Simply follow the on-screen
                               directions.  \\ \\
%
\underline{Edit/Delete Data Point} \hspace{1in} &
                               This button gives directions for
                               editing or deleting data.  For example,
                               to edit a data point, double-click on
                               the entry that you
                               want to edit. \\ \\

\underline{Add Point} &	       To add more data points to the
                               program, press this button and follow
                               the on-screen directions.  \\ \\


\underline{Switch X and Y}  &  Occasionally people inadvertently get
                               their X and Y measurements reversed
                               when they enter their data.  This
                               button allows one to switch all X and Y
                               values.  \\ \\


\underline{Clear Data in Program}  &	
                               If you want to do a fit to a new set of
                               data, press this button to clear away
                               all data currently in the program.\\ \\

\underline{Save Data to File}	&
                               This button prompts you to save your
                               data to a disk file always a good idea!
                               The button works exactly in the same way
                               as the Save Data button on the Home
                               screen.  \\ \\

\underline{Continue}  &  Press this button to return to the Home
                         screen. \\ \\


\end{tabular}

Some of these functions, and a few others, are also available from the
menu bar at the top of the screen.

\subsection*{Fitting Function Screen}

On this screen, choose the fitting function you want to use and enter
your estimates for your parameters, based on your analysis of your
graphs. Check the small box by any
parameter whose value you want to hold constant (that is, not allow to
vary in doing the fit).  {\bf Note:  }Only under unusual
circumstances will you ever need to hold a parameter constant

In introductory physics courses at CSB/SJU, we strongly encourage
students to do preliminary graphical analysis in their lab books and
in the process calculate reasonably accurate parameter estimates.
Only in this way can one develop an understanding of the data analysis
skills that are essential for scientists and engineers.  Please see
your instructor or consult the appropriate sections of your laboratory
manual if you are unsure how to do this preliminary analysis.


\

\subsection*{PGPlot Graph Screen}

The PGPlot screen gives you a plot of your data.  If you have already done
your fit, it also plots a graph of your fitting function , so that you can see how well the fitting function describes your data.  It also allows you to print graphs and to save them in PostScript format.

PGPlot is a set of Fortran routines that is available from the astronomy department at CalTech.  It is called as a separate program from Linfit.  Consequently, screen plots must be closed separately, by clicking the button in the top right of the graph window.

%Optionally, you may also plot the fitting function based on your
%estimates of the fit parameters, even before you have done a fit.  It is a good way to see how
%accurate your estimates are.

It is a good idea to look at a graph of your data BEFORE doing your
fit.  In this way you can get some idea of what fitting functions
might work.  For example, you would not want to try to fit a straight
line to data that did not lie along a straight line!  By doing
semi-log or log-log graphs, you can tell if your data are consistent
with an exponential or power law function.  And so on.

Note that the Graph screen allows you to set various options on the graph, such as axis
labels, size and style used for data points, and so on.  We encourage
you to experiment with these options.  On-line help is available.  Among other things,

\begin{itemize}

\item You can select a linear, semi-log ($\log y$ vs.\ $x$), or log-log
($\log y$ vs.\ $\log x$) graph.  Among other things, you can use this choice to
find out if your data are consistent with an exponential or power law fitting function before doing your fit.  (Recall that for an exponential function, a plot of log Y vs. X results in a straight line.  Similarly, for a power law function, a plot of log Y vs. log X results in a straight line.
Be sure you can confirm this behavior by taking the log of both sides of an exponential or power law function.
%
\item You can turn the error bars on or off.  Default is on.
%
\item If you have already done a fit, you have the option to plot
the
fitting function on top of your data.  Default is on.
%If you have not done a fit, but have entered your parameter estimates, you may plot the resulting fitting function.  Default is off for this case.  If your parameters are not reasonably accurate, the results are not always predictable.
Once you have selected your options, press the Screen Plot button to see your graph.  The Continue button will return you to the Home screen.
\item Note that the Graph Screen has several tabs that allow additional options for setting up graphs.
\end{itemize}

The buttons at the bottom on the Graph screen perform the following functions:
\begin{itemize}
\item The Screen Plot button produces an on-screen graph.  This graph will not close automatically; you need to close it manually by clicking on the close button at the top right of the graph window.
%
\item The Save PostScript File button saves the graph in PostScript format.  Such files can easily be converted to pdf format, and may also be examined in PostScript viewers such as GSView. 
\item The Print Graph (GhostScript) button will print the file on any Windows printer.  Note that the Print Options menu allows you to view the graph in GSView instead.  For more on these (free) programs, see http://pages.cs.wisc.edu/~ghost/.
\item The Print Graph (PostScript) button prints the graph to PostScript printers.  Most of the printers at CSB/SJU support PostScript.  However, if you use this button to print to a non-PostScript printer, you are likely to print many pages of PostScript code.
\item The Continue button returns you to the Home Screen.
\end{itemize}
Some of these options, and others as well, are available on the menu
bar for this screen.

\subsection*{Fit Screen}

This screen reports the results of the current fit.  To do the fit,
either use the menu Run/Go command or press the F5 function key.  You
can print the report from the File menu.  Note that you can also save the file in both pdf and Word (.rtf) formats.

If you have more than 20 or 25 data points, your report may be
divided into two or more pages.  Use the Up Arrow and Down Arrow
keyboard keys, the indicated function keys, or the appropriate menu
bar choices to move from one page to another.  The light blue toolbar just under the menus can be helpful in scrolling through your fit report.

\subsection*{Spreadsheet}

A spreadsheet can be a powerful tool for reducing data or doing any
sort of repetitive calculation.  You can usually use your calculator
for anything you might do in a spreadsheet; but many people find
spreadsheets more convenient.

The spreadsheet used in Linfit is to a considerable extent compatible
with Microsoft Excel.  It can read and write files in Excel format,
though it cannot use many of the
fancier features of Excel.  Nevertheless, to a considerable extent
work done in Excel can be read into Linfit, and vice versa.

We offer a few tips on how to use this spreadsheet here, but it is
impractical to give a complete description.  Perhaps the best way to
learn to use the spreadsheet is to read a book or take a workshop on
Microsoft Excel (or some similar spreadsheet).  The details of the
menu structure will not be the same.  But if you know the basics of
how to work with a spreadsheet, what it can do, how to enter formulas,
and the like, you should be able to work with the Linfit
spreadsheet with out much difficulty.

\paragraph*{Excel File Format} As of this writing (Summer 2009), we are introducing
a new spreadsheet module that should recognize the latest Excel file formats.

\paragraph*{How to enter formulas in the spreadsheet}
Select (that is, click on) a cell, type an equals sign (=) and enter
the formula.  The usual syntax applies: + - * / for add, subtract,
multiply, and divide respectively.  You may place a cell reference in
the formula by clicking on the cell.  You may use any of the functions
supported by the spreadsheet in the formula (see Selected Spreadsheet
Functions below).

\paragraph*{Spreadsheet Data Menu}

It is often convenient to move data to or from the data screen.  Most
of commands
on the spreadsheet Data Menu are specific to Linfit, and are designed
to make it as easy as possible to move data back and forth. These
commands include:

\begin{itemize}
\item {\em Get Data from Data Screen}\\
Use this menu item to transfer data from the Data Screen into the spreadsheet
%
\item {\em Get Fit Results from Fit Screen} \\
After you have done a least-squares fit, you can use this menu item to
transfer the fit results into the spreadsheet.
%
\item {\em Select Data} \\
You will often want to select data (one to three columns) before
moving these data to the Data Screen.  One can do so by hand, of
course: Simply drag the mouse over the range you want to select.  (You
can select non-contiguous areas of the spreadsheet by holding down the
Control (Ctrl) key as you drag the mouse.)

However, the Select Data option on the Data menu automates this
process.  To begin, select the columns (at least one, no more than
three) by clicking on the column (A, B, C, ...) at the top of the
spreadsheet.  To select more than one column, hold the Control (Ctrl)
key down as you select columns after the first one.  Then, choose
Data/Select Data.  You will be prompted to enter the beginning and
ending row numbers to be selected.
%
\item {\em Move Data to Data Screen} \\
Often you will want to move data from the spreadsheet to the Data
Screen in order to do a fit or make a graph.  To do so, select the
data either manually or choosing Select Data under the Data menu (see
above).  Then choose "Move Data to Data Screen" under the Data menu.
You will be prompted to tell the program whether the column or columns
you have selected represent X, Y, or Y Error values.
%
\item {\em Parse Text to Columns}
You can read a data file saved from a text editor or the Data screen
into the spreadsheet.  Often, however, there will only be spaces
between the columns of data (``space-delimited'' in computer terminology).
Both Excel and Linfit will read such data into a single column.

In order to put each column of data into a separate spreadsheet
column, select the data with the mouse or keyboard and choose "Parse
Text to Columns" from the Data menu.  \end{itemize}

\paragraph*{Selected spreadsheet functions}
Most spreadsheets contain a large collection of functions to perform
various tasks and calculations.  The following topics list some
available Linfit spreadsheet functions that are most likely to be
useful to scientists and engineers.  See the on-line help for details
on what these functions do, and how to use them.

This list is not complete---for
example, financial functions, date/time functions, and a good many
others are not included here.  If you happen to know such functions,
try them---they are likely to be here.

\paragraph*{Logical Functions}
\begin{center}
\begin{tabbing}	
TRUE \hspace{1in} \= FALSE \\
IF               \> OR   \\
NOT              \>       AND
\end{tabbing}
\end{center}
%
\paragraph*{Mathematical Functions}
\begin{tabbing}
ABS \hspace*{1in} \=  ACOS \hspace*{1in}  \= ACOSH \hspace*{1in} \=  ASIN \\
ASINH    \>  ATAN  \>  ATAN2  \>  ATANH \\
CHAR     \>  COS   \>  COSH   \>  EXP   \\
FACT     \>  INT   \>  LN     \>  LOG  \\
LOG10    \>  MAX   \>  MIN    \>  MOD  \\
PI       \>  ROUND \>  ROUNDDOWN   \>  ROUNDUP \\
SIGN     \>  SIN   \>  SINH   \>  SQRT   \\
STDEV    \>  STDEVP \> TAN    \>  TANH  \\
TRUNC   \\
\end{tabbing}

\paragraph*{Spreadsheet and Statistical Functions}
\begin{tabbing}	
AVERAGE  \hspace{1in}  \=  COLUMN \hspace{1in}  \=  ROW  \\
SUM     \>  SUMIF  \>  SUMSQ \\
STDEV
\end{tabbing}

\subsection*{LINFIT Troubleshooting}

Linfit is a linear least-squares fitting program that does
least-squares fits of experimental data to a variety of fitting
functions.  The bulk of the numerical analysis is done in Fortran code
based on algorithms adapted from Bevington, {\em Data Reduction}
\ldots,
and Press et al, {\em Numerical Recipes} (see References).  Linfit
employs
a Windows interface written in Microsoft Visual Basic; the Fortran
routines are compiled as Windows DLLs (Dynamic Link Libraries) and are
called from Visual Basic.

Although the program has been thoroughly tested and is reasonably
stable, there may still be some bugs.  Those bugs can be in the
programming, in Visual Basic, or even in Windows itself.  Please help
us by reporting any problems you happen to encounter.

If you can't get the program to fit your data or otherwise experience
difficulty, check your entries for error mode and fitting function
and see if there are any apparent problems.  For example, holding
a fitting function parameter constant can sometimes cause problems
if the value is zero or is otherwise inconsistent with your data.
Make sure your data file
is set up correctly, and that there are no ``nonsense" values.  If you
 can't run the program, the network may be down.  Ask people around
you---they will be having the same troubles if the network is down.  If all
else fails, and you think you may have discovered a bug in LINFIT,
please notify Professor Gearhart in the Physics Department (email:
cgearhart@csbsju.edu).  Please send along as much detailed information
as you can about what you were doing, how the program misbehaved, any
error messages you saw, and the like.





