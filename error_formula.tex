\newapp
\newpage
\section*{Appendix E---Uncertainty Formulae}
In the below equations %the quantities
$a,b,c,\ldots$ have uncertainties $\delta a,\delta b,\delta c,\ldots$ 
whereas %the quantities
$K, k_1,k_2,\ldots$ are ``constants" (like $\pi$ or 2) with
zero error or quantities with so small error that they can be treated
as error-free (like the mass of a proton: $m_p=(1.67262158\pm.00000013)\times 10^{-27}$~kg).
%or Avogadro's number $N_A = (6.022 141 99\pm .0000047)\times 10^{23}$).
This table reports the error in a calculated quantity $R$ (which is assumed to be positive).  
Note that a quantity like
$\delta a$ is called an {\em absolute} error; whereas the quantity
$\delta a/a$ is called the {\em relative} error (or, when multiplied by
100, the {\em percent} error).  The odd Pythagorean-theorem-like addition
(e.g., $\delta R= \sqrt{\delta a^2 + \delta b^2}$) is called ``addition in
quadrature".  Thus the formula for the error in $R=K\;{ab/cd}$ could be stated as
``the percent error in $R$ is the sum of the percent errors in $a,b,c$ and $d$ added
in quadrature".\label{eq:E.error}

\begin{center}
\begin{tabular}{lclr}      
{\Large Equation}&$\quad$&{\Large Uncertainty}\\ \\
${\displaystyle R= a+K }$&&
${\displaystyle \delta R= \delta a
}$&(E.1)\\ \\
${\displaystyle R= K\; a }$&&
${\displaystyle {\delta R \over R}={\delta a \over |a|}
}\quad$ or $\quad{\displaystyle {\delta R }={|K|\; \delta a }
}$&(E.2)\\ \\
${\displaystyle R= {K\over a} }$&&
${\displaystyle {\delta R \over R}={\delta a \over |a|}
}\quad$ or $\quad{\displaystyle {\delta R }={|K|\; \delta a \over a^2}
}$&(E.3)\\ \\
${\displaystyle R= K\; a^{k_1} }$&&
${\displaystyle {\delta R \over R}={|k_1|\;\delta a \over |a|}
}\quad$ or $\quad{\displaystyle {\delta R }={|k_1K a^{k_1-1}|\; \delta a }
}$&(E.4)\\ \\
${\displaystyle R= a\pm b }$&&
${\displaystyle \delta R= \sqrt{\delta a^2 + \delta b^2} 
}$&(E.5)\\ \\
${\displaystyle R= k_1 a + k_2 b }$&&
${\displaystyle \delta R=\sqrt{
\left({k_1 \delta a}\right)^2+
\left({k_2 \delta b}\right)^2}
}$&(E.6)\\ \\
${\displaystyle R= K\;ab }$&&
${\displaystyle {\delta R \over R}=\sqrt{
\left({\delta a\over a}\right)^2+
\left({\delta b\over b}\right)^2}
}$&(E.7)\\ \\
${\displaystyle R= K\;{a\over b} }$&&
${\displaystyle {\delta R \over R}=\sqrt{
\left({\delta a\over a}\right)^2+
\left({\delta b\over b}\right)^2}
}$&(E.8)\\ \\
${\displaystyle R= f(a) }$&&
${\displaystyle \delta R= |f'(a)|\;\delta a
}$&(E.9)\\ \\
${\displaystyle R=K\;{ab \over cd} }$&&
${\displaystyle {\delta R \over R}=\sqrt{
\left({\delta a\over a}\right)^2+
\left({\delta b\over b}\right)^2+
\left({\delta c\over c}\right)^2+
\left({\delta d\over d}\right)^2} }$&(E.10)\\ \\
${\displaystyle R=K\;{a^{k_1}b^{k_2} \over c^{k_3}d^{k_4}} }$&&
${\displaystyle {\delta R \over R}=\sqrt{
\left({k_1 \delta a\over a}\right)^2+
\left({k_2 \delta b\over b}\right)^2+
\left({k_3 \delta c\over c}\right)^2+
\left({k_4 \delta d\over d}\right)^2} }$&(E.11)\\ \\
${\displaystyle R=f(a,b,c,d) }$&&
${\displaystyle \delta R=\sqrt{
\left({\partial f \over \partial a}\;\delta a\right)^2+
\left({\partial f \over \partial b}\;\delta b\right)^2+
\left({\partial f \over \partial c}\;\delta c\right)^2+
\left({\partial f \over \partial d}\;\delta d\right)^2} }$&(E.12)\\
\end{tabular}
\end{center}
