%SIMPLE HARMONIC MOTION
\newexp
\section*{Introduction}
     In this experiment we will examine the behavior of a simple 
harmonic oscillator, that is, a mass attached to the end of a 
spring.  Remarkably, this system is one of the most important 
in all of physics.  Almost any problem that involves 
oscillations --- for example, the electrical oscillations in a 
radio, or the oscillations of an atom around a lattice site in a 
crystal --- can at least be approximated by the simple harmonic 
oscillator.

\section*{Analysis}
     As you may know from the lectures and textbook, Hooke's law 
states that the force exerted by a spring on a mass is given by
\begin{equation}
F = - kx   \label{eq:harm1}
\end{equation}
where $x$ is the displacement of the spring from its equilibrium 
position and $k$, called the spring constant, is a constant 
of proportionality that depends on the stiffness of the spring.

     We will attempt to measure the spring constant in two ways.  
In the first part of the experiment, we will hang masses on the 
spring, and measure the displacement from equilibrium for a 
number of different masses.  When the mass is at rest the 
gravitational force $mg$ will just balance the force exerted by the 
spring.  Hence a graph of force vs.\ displacement should allow you 
either to confirm or reject Eq.~\ref{eq:harm1}, and if the graph is a 
straight line, to determine the value of $k$ from a computer fit.

     In the second part of the experiment we will set the mass 
oscillating, and measure the period of oscillation, $T$ (that is, the 
time for one complete oscillation) as a function of mass.  Then, 
you can use your knowledge of graphical and computer analysis to 
find the functional relationship between period and mass.  Look 
up the theoretical expression for this relationship and use it 
along with your computer analysis to find a value for the spring 
constant $k$.  See if $k$ determined in this way agrees, within the 
limits of experimental uncertainty, with the value you found in 
the first part of the experiment.

\section*{Experimental Procedure}
     Examine the apparatus carefully and work out a suitable 
procedure for the experiment.  Keep in mind the following points:
\begin{enumerate}
\item Be sure you measure the mass of the weight holder, and add
it to your experimental masses.  You need the {\bf total} mass in
motion.
%calculate the force on the spring!

\item Be sure you adjust the wire pointer on the apparatus to 
         eliminate the effects of parallax.
\item The largest source of experimental uncertainty in the 
         measurement of the period is in deciding exactly when to 
         start your timer, and exactly when to stop it.  The 
         effect of this error can be minimized by measuring the 
         time for a large number of oscillations.
\end{enumerate}
    
\section*{Analysis of Data}
     {\bf NOTE:}  Please be sure that you know what your units are, and
     that you are using units consistently, in the following
     analysis.

     Use \WAPP to find:
\begin{enumerate}     
\item The value of the spring constant $k$, using Hooke's law.  Be sure
that you also report the uncertainty in $k$, based on your least-squares
fit.  Print out a fit report and  plot.

%\item The functional dependence of the period of oscillation 
%         on the mass.  Check your textbook, and compare your
%	   result to the theoretical result. 

%change in method: found T^2 vs m plot (while good since m_0 inconsequencial)
%generally results in too small slope (due to difficulty of high frequency m=100g case?)
% Hooke's law ks are consistent (\pm1 N/m) whereas oscillation k are
%divergent and generally large +20%
%solution(?): try power law approach
 
\item The value of the spring constant $k$, using your 
         oscillation period data. According to our work 
in class, the period has a power law relationship
to the oscillating mass:
\[ T={2\pi\over\sqrt{k}}\; m^{1/2} \]

First, look at a graph of log (period) vs. log (mass).  If it is linear, fit your period vs.\ mass data to a power law relationship and compare to the theoretical result (see your textbook).
Print out fit report and  plot.
Calculate and report the uncertainty in $k$, based on your least-squares
fit.

\item See if your two values 
         of the spring constant agree within the limits of 
         experimental uncertainty.  
%{\bf Note:}  This part is tricky
%and you may need to consult your instructor.  Don't leave this
%part until the last moment!
\end{enumerate}

%\section*{Questions}
%\begin{enumerate}
%\item How could you find the velocity of the oscillating mass 
%         at any arbitrary value of the displacement $x$?
%\end{enumerate}

\section*{Conclusions}
Make a table listing both of your $k$ results (including uncertainties and the
value of reduced $\chi^2$ of the corresponding fit).
Comment on the consistency of your results for the determination of $k$ from
the two different methods.  Mention any sources of error, both
systematic and random.  Attach both fit results and plots.

\section*{Quick Report Card}
Properly report (sigfigs, units, error) your two values for $k$.

\section*{Critique of Lab}
     Follow the suggestions given in the Introduction to the
Laboratory Manual.
