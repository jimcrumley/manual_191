% INTRODUCTION
\newapp
\section*{Purpose}

\begin{quote}
When you measure what you are speaking about, and express it in numbers,
you know something about it, but when you cannot express it in numbers,
your knowledge is of a meager and unsatisfactory kind: It may be the beginning 
of knowledge, but you have scarcely in your thoughts advanced to
the stage of science.  {\em Lord Kelvin}
\end{quote}


Physics and engineering rely on quantitative experiments.
Experiments are artificial simplifications of nature:
line drawings rather than color photographs.  The hope is that by striping away
the details, the essence of nature is revealed.  
Although the aim of experiment is appropriate 
simplification, the design of experiments is anything but simple.
Typically it involves days (weeks, months, \ldots) of ``fiddling"
before the experiment finally ``works".  I wish this sort of creative
problem-oriented process
could be taught in a scheduled lab period to incoming students,
but limited time and the many prerequisites make this impossible.
(You have to work on grammar before you can write the great American
novel.)  Look for more creative labs starting next year!

%Rejecting magic, we assume that nature is consistent, and further that
%we are equipped to discover the (mathematical) rules
%that nature follows.  (The critics of science would note the hubris
%is this statement.)  

Thus this lab manual describes experiences (``labs")
that are caricatures of experimental physics.
Our labs will typically 
emphasize thorough preparation, an underlying mathematical
model of nature, good experimental technique, analysis
of data (including the significance of error)
\ldots the basic prerequisites for doing science.
But your creativity will be circumscribed.  You will find here ``instructions"
that are not a part of real experiments (where the methods and/or  outcomes
are not known in advance).  

The goals of these labs are therefore limited.  You will:
\begin{enumerate}
\item Perform certain experiments that illustrate the foundations of 
Newton's mechanics. 
\item Perform basic measurements and recognize the associated limitations
(which, when expressed as a number, are called uncertainties or errors).
\item Practice the methods which allow you to determine how uncertainties 
in measured quantities propagate to produce uncertainties 
in calculated quantities.
\item Practice the process of verifying a mathematical model,
including data collection, data display, and data analysis (particularly
graphical data analysis with curve fitting).
\item Practice the process of keeping an adequate lab notebook.
\item Experience the process of ``fiddling" with an experiment until it finally
``works".
\item Develop an appreciation for the highs and lows of lab work.
And I hope: learn to learn from the lows.
\end{enumerate}

\section*{Semester Lab Schedule}
To be announced in class.  Note that this Manual does not list
labs in the scheduled order.

You should be enrolled in a lab section for PHYS 191.  Your labs will
be completed on the cycle day/time for which you have enrolled.  These are
the only times that the equipment and assistance will be available to
you.  It is your responsibility to show up and complete each
lab.  (Problems meeting the schedule should be 
addressed---well in advance---to the lab manager.)

\section*{Materials}
You should bring the following to each lab:
\begin{itemize}
\item Lab notebook.  You will need three notebooks: While one is being
graded, the others will be available to use in the following labs.  The lab
notebook should have quad-ruled paper (so that it can be used for
graphs) and a sewn binding (for example, Ampad \#26--251, available in
the campus bookstores).
%
  \item Lab Manual (this one)
%
  \item The knowledge you gained from carefully reading the lab manual
  before you attended.
  \item A calculator, preferably scientific.
  \item A straightedge (for example, a 6" ruler).
\item A pen (we strongly prefer your lab book be written in
ink, since one should {\bf never}, under any circumstances, erase).
However, pencils are ok for drawings and graphs.
%  \item A 3.5" floppy disk to save your data.  (However, see the
%  last section of the Introduction for other options for saving
%  your data.)
\end{itemize}

%\section*{Open Lab System}
\section*{Before Lab:}

     Since you have a limited time to use equipment (other
students will need it), it will be to your advantage
if you come to the laboratory well prepared.  Please read the
description of the experiment carefully, and do any preliminary work
{\em in your lab notebook} before you come to lab.   (The section below
headed ``Lab Notebook'' gives a more complete description of what to
include in your lab notebook.)

There may be one or more quizzes
sometime during the semester to test how well you prepared for the lab.
These quizzes, as well as the pre-lab assignments, will be collected during
the FIRST TEN MINUTES of lab, so it is wise to show up on time or early.

\section*{During Lab:}

Note the condition of
your lab station when you start so that you can return it to that state when you leave.
Check the apparatus assigned to you.  Be sure you know the function of
each piece of equipment and that all the required pieces are present.
If you have questions, ask your instructor.  Usually you will want
to make a sketch of the setup in your notebook.
Prepare your
experimental setup and decide on a procedure to follow in
collecting data.  
Keep a running outline in your notebook of the procedure actually used.
If the procedure used is identical to that in this Manual,
you need only note ``see Manual".  Nevertheless, an outline of your
procedure can be useful even if you aim to exactly follow the Manual.
Prepare
tables for recording data (leave room for calculated quantities).
{\em Write your data in your notebook as you collect it!}  

Check your data table and graph, and make sample calculations if
pertinent to see if everything looks satisfactory before going on
to something else.  Most physical quantities will appear to vary
continuously and thus yield a smooth curve.  If your data looks
questionable (e.g., a jagged, discontinuous ``curve") you should take
some more data near the points in question.  Check with the instructor 
if you have any doubts.

Complete the analysis of data in your notebook and
 indicate your final results clearly.  If you make repeated calculations
 of any quantity, you need only show one sample calculation.  
Often a spreadsheet will be used to make repeated calculations.
In this case it is particularly important to report how each column was
calculated.  
Tape computer-generated data tables, plots and least-squares fit
reports into your notebook, so that they can be examined
easily.  Answer all questions that were asked in the Lab Manual.

{\bf CAUTION:}  {\em for your protection and for the good of the
equipment, please check with the instructor before turning on
any electrical devices.}

 \section*{Lab Notebook}

Your lab notebook should represent a {\em detailed} record of what
you have done in the laboratory.   It should be complete enough
so that you could look back on this notebook after a year or two and
reconstruct your work.    Sample notebook
pages from a previous student are included at the end of this
section.

     Your notebook should include your preparation for lab,
sketches and diagrams to explain the experiment, data collected,
initial graphs (done as data is collected), comments on
difficulties, sample calculations, data analysis, final graphs,
results, and answers to questions asked in the lab manual.  {\bf
NEVER} delete, erase, or tear out sections of your notebook that
you want to change.  Instead, indicate in the notebook what you
want to change and why (such information can be valuable later on).
Then lightly draw a line through the unwanted section and proceed with
the new work.

{\bf DO NOT} collect data or other information on other sheets
of paper and then transfer to your notebook.  Your notebook
is to be a running record of what you have done, not a formal
(all errors eliminated) report.  There will be no formal lab
reports in this course.  When you have finished a particular lab, you turn in
your notebook at the end of the period.

Ordinarily, your notebook should include the following items for each
experiment.

\begin{description}
\item {\bf NAMES}.  The title of the experiment, your name, your
lab partner's name, and your lab station number.

\item {\bf DATES}.  The date the experiment was performed.

\item {\bf PURPOSE}.  A brief statement of the objective or purpose of the
experiment.

\item {\bf THEORY}.  At least a listing of the relevant equations, and what the symbols
represent.  Often it is useful to number these equations so you can
unambiguously refer to them.

{\bf Note:}  These first four items can usually be completed before
you come to lab.  

\item {\bf PROCEDURE}.  This section should be an outline of
what you did in lab.  
As an absolute minimum your procedure must clearly describe
the data.  For example, a column of numbers labeled ``voltage"
is not sufficient.  You must identify how the voltage was measured,
the scale settings on the voltmeter, etc.  Your diagram of the
apparatus (see below) is usually a critical part of this
description, as it is usually easier to draw how the data
were measured than describe it in words.  Sometimes
your procedure will be identical to that
described in the lab manual, and it's OK to say simply so.
However there are usually
details you can fill in about the procedure.  Your procedure
may have been different from that described in the lab manual.
Or points that seem important to you may not have been included.
And so on.  This section is also a good place to describe any
difficulties you encountered in getting the experiment set up and
working.

\item {\bf DIAGRAMS}.  A sketch of the
apparatus is almost always required.  A simple block diagram can often describe the
experiment better than a great deal of written explanation.

\item {\bf DATA}.  You should record {\em all} the numbers
you encounter,
including {\em units} and {\em uncertainties}, and any other relevant
observations  as the experiment progresses.  The lab
report should include the {\em actual} data taken in the lab, not
recopied versions.  If you find it difficult to be neat and organized while
the experiment is in progress, you might try using the left-hand pages of
your notebook for doodles, raw data, rough calculations, etc., and later
transfer the important items to the right-hand pages.

This section can also include computer-generated data tables and
Linfit reports if
appropriate---just tape them into your lab book (one per page please).

It's good practice to graph the data as you acquire
it in the lab; this practice allows you to see where more data are
needed
and whether some measurements are suspicious and should be repeated.

\item {\bf CALCULATIONS}.  {\bf Sample calculations} should be included to
show how results are obtained from the data, and how the uncertainties
in the {\em results} are related to the uncertainties in the {\em
data} (see Appendix A).  For example, if you calculate the slope and
intercept of a straight line on a graph, you should show your work
in detail.  Your TA must be able to reproduce your every calculation
just based on your notebook.  The TAs have been instructed to
totally disregard answers that appear without an obvious source.
It is particularly important to remember to show how each column in a
spreadsheet hardcopy was calculated.

\item {\bf RESULTS/CONCLUSIONS}. You should end each experiment with a conclusion that
summarizes your results --- how successful was the experiment, what did
you learn from it, and what were your results (including numerical
results, with experimental errors).

  This section should begin with a
summary of your results, collected in a {\em carefully constructed}
table that summarizes all of your results in one place. Always include
units and experimental error.

You should also compare your results to the theoretical and/or
accepted values.
Does your experimental range of uncertainty overlap the accepted value?
Based on your results, what does the experiment tell you?

\item {\bf DISCUSSION/CRITIQUE}.
As a service to us and future students we would appreciate it if you
would also include a short critique of the lab in your notebook.
  Please
comment on such things as the clarity of the Lab Manual, performance
of equipment, relevance of experiment, and if there is anything you
particularly liked or disliked about the lab.  This is a good place to
blow off a little steam.  Don't worry; you won't be penalized, and we
use constructive criticisms to help improve these experiments.

%\item {\bf QUICK REPORT}.
%As you leave lab, turn in a $3\times5$ ``quick report" card.
%You will be told in lab what information belongs on your card.
%These cards go directly to the Lab Manager who will use them to
%identify problems.

\end{description}

%At the end of most labs in this manual there is a checklist for you to
%use when finishing up the lab.  {\em HINT: it's the same checklist
%your TA will use when grading your work!} It can serve as a guide for
%you so you won't forget some important part of the experiment.  You'll
%usually find an item called ``Cleanup" in each checklist.  Your lab TA
%will check your lab station when you leave.  Equipment, hookups, and
%everything else should appear exactly as it did when you sat down to
%start the lab.

Drop off your Lab Notebook in your Lab Instructor's box. 
Note: {\em The TA's cannot accept late labs. If for some reason you
cannot complete a lab on time, please see the lab manager (Lynn Schultz) in EngelSC 139 or call
363--2835. Late labs will only be accepted under exceptional circumstances.
If an exception is valid, the lab may still be penalized depending on how
responsibly you handled the situation (e.g., did you call BEFORE the lab 
started?).}

\section*{Grading}

Each lab in your notebook will be graded separately as follows: \\
\begin{tabbing}
1234567890123456 \= col2 \kill
9--10 points: \> A \\
8--8.9 points: \> B \\
7--7.9 points: \> C \\
6--6.9 points: \> D \\
0--5.9 points: \> Unsatisfactory \\
\end{tabbing}


%\newpage
%\vspace*{\fill}
\begin{figure}[hbt]
   \centerbmp{6.81in}{9in}{sample.bmp}
 %\caption{CAPTION HERE \label{fig:REFERENCE HERE}}
\end{figure}
\clearpage
%\newpage
%\vspace*{\fill}
\begin{figure}[hbt]
   \centerbmp{6.81in}{9in}{sample2.bmp}
   %\special{bmp:sample2.bmp}
 %\caption{CAPTION HERE \label{fig:REFERENCE HERE}}
\end{figure}
%\newpage
\clearpage

