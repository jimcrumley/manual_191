% Kater Pendulum
\newexp

\section*{Introduction}

It is well-known result that the period T of a simple pendulum is given by
% MathType!MTEF!2!1!+-
% feaaeaart1ev0aaatCvAUfKttLearuavP1wzZbItLDhis9wBH5garm
% Wu51MyVXgaruWqVvNCPvMCG4uz3bWemv3yPrwynfgDOLeDHXwAJbqe
% gWuDJLgzHbIqYL2zOrhinfgDObss0fgBPngarqqtubsr4rNCHbGeaG
% qiVCI8FfYJH8sipiYdHaVhbbf9v8qqaqFr0xc9pk0xbba9q8WqFfea
% Y-biLkVcLq-JHqpepeea0-as0Fb9pgeaYRXxe9vr0-vr0-vqpWqaae
% aabiGaciaacaqabeaadaabauaaaOqaaiabdsfaujabg2da9iabikda
% Yiabec8aWnaakaaabaWaaSaaaeaacqWGmbataeaacqWGNbWzaaaale
% qaaaaa!46A8!
\[
T = 2\pi \sqrt {\frac{L}{g}} 
\]
where $L$ is the length.  In principle, then, a pendulum could be used
to measure $g$,  the acceleration of gravity.  However, practical
difficulties-primarily in measuring the length accurately-make it
unsatisfactory for high precision measurements.

It turns out that a physical pendulum-typically, a mass suspended from
a knife edge-can be used for much more accurate measurements.  The
Kater pendulum is such a physical pendulum.  It is named after its
inventor, Captain Henry Kater, a captain in the British army and a
Fellow of the Royal Society, who invented it around 1820.  For many
years, it was the standard method of measuring $g$ to high accuracy.

\begin{figure}[!hbt]     %Kater pendulum drawing, kater.eps
\vspace{3in}
\special{eps:kater.eps}
\caption{Kater Pendulum.  The pendulum can oscillate about a knife edge through either hole.  The holes are located so that the two periods of oscillation are nearly equal. \label{fig:kater}}
\end{figure}

The beauty of this experiment is that it allows us, using relatively
simple apparatus, to measure a physical quantity very accurately
indeed-we hope, to within a few parts in 10,000.  It works as follows:
Consider a physical pendulum with two supports that lie along a line
through the center of mass, as shown in Figure 1.

Suppose the distances $d_1$ and $d_2$ are adjusted so that the periods
around both supports are equal: 
$%\[
T_1  = T_2  = T
$%\]
  .  Then it can be shown that this
period is the period of a simple pendulum with length $L$ , the distance
between the supports.  Thus, if the period and that distance can both
be measured accurately, one can find the acceleration of gravity:
% MathType!MTEF!2!1!+-
% feaaeaart1ev0aaatCvAUfKttLearuavP1wzZbItLDhis9wBH5garm
% Wu51MyVXgaruWqVvNCPvMCG4uz3bWemv3yPrwynfgDOLeDHXwAJbqe
% gWuDJLgzHbIqYL2zOrhinfgDObss0fgBPngarqqtubsr4rNCHbGeaG
% qiVCI8FfYJH8sipiYdHaVhbbf9v8qqaqFr0xc9pk0xbba9q8WqFfea
% Y-biLkVcLq-JHqpepeea0-as0Fb9pgeaYRXxe9vr0-vr0-vqpWqaae
% aabiGaciaacaqabeaadaabauaaaOqaaiabdsfaujabg2da9iabikda
% Yiabec8aWnaakaaabaWaaSaaaeaacqWGKbazdaWgaaWcbaGaeGymae
% dabeaakiabgUcaRiabdsgaKnaaBaaaleaacqaIYaGmaeqaaaGcbaGa
% em4zaCgaaaWcbeaakiaaywW7cqqGVbWBcqqGYbGCcaaMf8Uaeeiiaa
% Iaem4zaCMaeyypa0ZaaSaaaeaacqaI0aancqaHapaCdaahaaWcbeqa
% aiabikdaYaaakmaabmaabaGaemizaq2aaSbaaSqaaiabigdaXaqaba
% GccqGHRaWkcqWGKbazdaWgaaWcbaGaeGOmaidabeaaaOGaayjkaiaa
% wMcaaaqaaiabdsfaunaaCaaaleqabaGaeGOmaidaaaaaaaa!620A!
\begin{equation}%\[
T = 2\pi \sqrt {\frac{{d_1  + d_2 }}{g}} \quad {\rm or}\quad {\rm  }g = \frac{{4\pi ^2 \left( {d_1  + d_2 } \right)}}{{T^2 }}
 \label{eq:kater1}
\end{equation}%\]

\section*{Theory}

In this section, we give a detailed derivation of Equation~(\ref{eq:kater1}).  The period of a physical pendulum is

% MathType!MTEF!2!1!+-
% feaaeaart1ev0aaatCvAUfKttLearuavP1wzZbItLDhis9wBH5garm
% Wu51MyVXgaruWqVvNCPvMCG4uz3bWemv3yPrwynfgDOLeDHXwAJbqe
% gWuDJLgzHbIqYL2zOrhinfgDObss0fgBPngarqqtubsr4rNCHbGeaG
% qiVCI8FfYJH8sipiYdHaVhbbf9v8qqaqFr0xc9pk0xbba9q8WqFfea
% Y-biLkVcLq-JHqpepeea0-as0Fb9pgeaYRXxe9vr0-vr0-vqpWqaae
% aabiGaciaacaqabeaadaabauaaaOqaaiabdsfaujabg2da9iabikda
% Yiabec8aWnaakaaabaWaaSaaaeaacqWGjbqsaeaacqWGnbqtcqWGNb
% WzcqWGKbazaaaaleqaaaaa!4916!
\[
T = 2\pi \sqrt {\frac{I}{{Mgd}}} 
\]

where $I$ is the moment of inertia about the axis of rotation 
of the pendulum, and $d$ is the distance from the axis of rotation to the center of mass.  From the Parallel Axis Theorem, it is apparent 
that

% MathType!MTEF!2!1!+-
% feaaeaart1ev0aaatCvAUfKttLearuavP1wzZbItLDhis9wBH5garm
% Wu51MyVXgaruWqVvNCPvMCG4uz3bWemv3yPrwynfgDOLeDHXwAJbqe
% gWuDJLgzHbIqYL2zOrhinfgDObss0fgBPngarqqtubsr4rNCHbGeaG
% qiVCI8FfYJH8sipiYdHaVhbbf9v8qqaqFr0xc9pk0xbba9q8WqFfea
% Y-biLkVcLq-JHqpepeea0-as0Fb9pgeaYRXxe9vr0-vr0-vqpWqaae
% aabiGaciaacaqabeaadaabauaaaOqaaiabdMeajjabg2da9iabdMea
% jnaaBaaaleaacqWGJbWyaeqaaOGaey4kaSIaemyta0Kaemizaq2aaW
% baaSqabeaacqaIYaGmaaaaaa!4855!
\[
I = I_c  + Md^2 
\]
where $I_c$ is the moment of inertia about the center of mass.  
Thus for our physical pendulum in Figure 1, we have

% MathType!MTEF!2!1!+-
% feaaeaart1ev0aaatCvAUfKttLearuavP1wzZbItLDhis9wBH5garm
% Wu51MyVXgaruWqVvNCPvMCG4uz3bWemv3yPrwynfgDOLeDHXwAJbqe
% gWuDJLgzHbIqYL2zOrhinfgDObss0fgBPngarqqtubsr4rNCHbGeaG
% qiVCI8FfYJH8sipiYdHaVhbbf9v8qqaqFr0xc9pk0xbba9q8WqFfea
% Y-biLkVcLq-JHqpepeea0-as0Fb9pgeaYRXxe9vr0-vr0-vqpWqaae
% aabiGaciaacaqabeaadaabauaaaOqaaiabdsfaunaaBaaaleaacqaI
% XaqmaeqaaOGaeyypa0JaeGOmaiJaeqiWda3aaOaaaeaadaWcaaqaai
% abdMeajnaaBaaaleaacqWGJbWyaeqaaOGaey4kaSIaemyta0Kaemiz
% aq2aa0baaSqaaiabigdaXaqaaiabikdaYaaaaOqaaiabd2eanjabdE
% gaNjabdsgaKnaaBaaaleaacqaIXaqmaeqaaaaaaeqaaOGaeeiiaaIa
% eeiiaaIaeeiiaaIaaGzbVlabbggaHjabb6gaUjabbsgaKjabbccaGi
% aaywW7cqqGGaaicqqGGaaicqWGubavdaWgaaWcbaGaeGOmaidabeaa
% kiabg2da9iabikdaYiabec8aWnaakaaabaWaaSaaaeaacqWGjbqsda
% WgaaWcbaGaem4yamgabeaakiabgUcaRiabd2eanjabdsgaKnaaDaaa
% leaacqaIYaGmaeaacqaIYaGmaaaakeaacqWGnbqtcqWGNbWzcqWGKb
% azdaWgaaWcbaGaeGOmaidabeaaaaaabeaaaaa!7134!
\[
T_1  = 2\pi \sqrt {\frac{{I_c  + Md_1^2 }}{{Mgd_1 }}} {\rm    }\quad {\rm and }\quad {\rm   }T_2  = 2\pi \sqrt {\frac{{I_c  + Md_2^2 }}{{Mgd_2 }}} 
\]

But we can always write $I_c$ in terms of the radius of 
gyration $k$, the radius of a cylindrical ring that has the 
same mass $M$ and the same rotational inertia $I$ as the actual 
object:

% MathType!MTEF!2!1!+-
% feaaeaart1ev0aaatCvAUfKttLearuavP1wzZbItLDhis9wBH5garm
% Wu51MyVXgaruWqVvNCPvMCG4uz3bWemv3yPrwynfgDOLeDHXwAJbqe
% gWuDJLgzHbIqYL2zOrhinfgDObss0fgBPngarqqtubsr4rNCHbGeaG
% qiVCI8FfYJH8sipiYdHaVhbbf9v8qqaqFr0xc9pk0xbba9q8WqFfea
% Y-biLkVcLq-JHqpepeea0-as0Fb9pgeaYRXxe9vr0-vr0-vqpWqaae
% aabiGaciaacaqabeaadaabauaaaOqaaiabdMeajnaaBaaaleaacqWG
% JbWyaeqaaOGaeyypa0Jaemyta0KaaGjcVlabdUgaRnaaCaaaleqaba
% GaeGOmaidaaaaa!47F7!
\[
I_c  = M{\kern 1pt} k^2 
\]

Hence the equation for the periods can be written

% MathType!MTEF!2!1!+-
% feaaeaart1ev0aaatCvAUfKttLearuavP1wzZbItLDhis9wBH5garm
% Wu51MyVXgaruWqVvNCPvMCG4uz3bWemv3yPrwynfgDOLeDHXwAJbqe
% gWuDJLgzHbIqYL2zOrhinfgDObss0fgBPngarqqtubsr4rNCHbGeaG
% qiVCI8FfYJH8sipiYdHaVhbbf9v8qqaqFr0xc9pk0xbba9q8WqFfea
% Y-biLkVcLq-JHqpepeea0-as0Fb9pgeaYRXxe9vr0-vr0-vqpWqaae
% aabiGaciaacaqabeaadaabauaaaOqaaiabdsfaunaaBaaaleaacqaI
% XaqmaeqaaOGaeyypa0JaeGOmaiJaeqiWda3aaOaaaeaadaWcaaqaai
% abd2eanjabdUgaRnaaCaaaleqabaGaeGOmaidaaOGaey4kaSIaemyt
% a0Kaemizaq2aa0baaSqaaiabigdaXaqaaiabikdaYaaaaOqaaiabd2
% eanjabdEgaNjabdsgaKnaaBaaaleaacqaIXaqmaeqaaaaaaeqaaOGa
% eeiiaaIaeeiiaaIaaGzbVlabbccaGiabbggaHjabb6gaUjabbsgaKj
% abbccaGiaaywW7cqqGGaaicqqGGaaicqWGubavdaWgaaWcbaGaeGOm
% aidabeaakiabg2da9iabikdaYiabec8aWnaakaaabaWaaSaaaeaacq
% WGnbqtcqWGRbWAdaahaaWcbeqaaiabikdaYaaakiabgUcaRiabd2ea
% njabdsgaKnaaDaaaleaacqaIYaGmaeaacqaIYaGmaaaakeaacqWGnb
% qtcqWGNbWzcqWGKbazdaWgaaWcbaGaeGOmaidabeaaaaaabeaaaaa!734A!
\[
T_1  = 2\pi \sqrt {\frac{{Mk^2  + Md_1^2 }}{{Mgd_1 }}} {\rm   }\quad {\rm  and }\quad {\rm   }T_2  = 2\pi \sqrt {\frac{{Mk^2  + Md_2^2 }}{{Mgd_2 }}} 
\]

or simplifying,

% MathType!MTEF!2!1!+-
% feaaeaart1ev0aaatCvAUfKttLearuavP1wzZbItLDhis9wBH5garm
% Wu51MyVXgaruWqVvNCPvMCG4uz3bWemv3yPrwynfgDOLeDHXwAJbqe
% gWuDJLgzHbIqYL2zOrhinfgDObss0fgBPngarqqtubsr4rNCHbGeaG
% qiVCI8FfYJH8sipiYdHaVhbbf9v8qqaqFr0xc9pk0xbba9q8WqFfea
% Y-biLkVcLq-JHqpepeea0-as0Fb9pgeaYRXxe9vr0-vr0-vqpWqaae
% aabiGaciaacaqabeaadaabauaaaOqaaiabdsfaunaaBaaaleaacqaI
% XaqmaeqaaOGaeyypa0JaeGOmaiJaeqiWda3aaOaaaeaadaWcaaqaai
% abdUgaRnaaCaaaleqabaGaeGOmaidaaOGaey4kaSIaemizaq2aa0ba
% aSqaaiabigdaXaqaaiabikdaYaaaaOqaaiabdEgaNjabdsgaKnaaBa
% aaleaacqaIXaqmaeqaaaaaaeqaaOGaeeiiaaIaeeiiaaIaaGzbVlab
% bccaGiabbggaHjabb6gaUjabbsgaKjabbccaGiabbccaGiaaywW7cq
% qGGaaicqWGubavdaWgaaWcbaGaeGOmaidabeaakiabg2da9iabikda
% Yiabec8aWnaakaaabaWaaSaaaeaacqWGRbWAdaahaaWcbeqaaiabik
% daYaaakiabgUcaRiabdsgaKnaaDaaaleaacqaIYaGmaeaacqaIYaGm
% aaaakeaacqWGNbWzcqWGKbazdaWgaaWcbaGaeGOmaidabeaaaaaabe
% aaaaa!6C78!
\begin{equation}%\[
T_1  = 2\pi \sqrt {\frac{{k^2  + d_1^2 }}{{gd_1 }}} {\rm   }\quad {\rm  and  }\quad {\rm  }T_2  = 2\pi \sqrt {\frac{{k^2  + d_2^2 }}{{gd_2 }}} 
 \label{eq:bothT}
\end{equation}%\]
(Note that by using the radius of gyration, we have eliminated the
mass.  This step is not necessary, but it makes the rest of the
derivation less cumbersome.)

If we have adjusted the distances $d_1$ and $d_2$ so that the periods are
the same, it follows that

% MathType!MTEF!2!1!+-
% feaaeaart1ev0aaatCvAUfKttLearuavP1wzZbItLDhis9wBH5garm
% Wu51MyVXgaruWqVvNCPvMCG4uz3bWemv3yPrwynfgDOLeDHXwAJbqe
% gWuDJLgzHbIqYL2zOrhinfgDObss0fgBPngarqqtubsr4rNCHbGeaG
% qiVCI8FfYJH8sipiYdHaVhbbf9v8qqaqFr0xc9pk0xbba9q8WqFfea
% Y-biLkVcLq-JHqpepeea0-as0Fb9pgeaYRXxe9vr0-vr0-vqpWqaae
% aabiGaciaacaqabeaadaabauaaaOqaamaalaaabaGaem4AaS2aaWba
% aSqabeaacqaIYaGmaaGccqGHRaWkcqWGKbazdaqhaaWcbaGaeGymae
% dabaGaeGOmaidaaaGcbaGaemizaq2aaSbaaSqaaiabigdaXaqabaaa
% aOGaeyypa0ZaaSaaaeaacqWGRbWAdaahaaWcbeqaaiabikdaYaaaki
% abgUcaRiabdsgaKnaaDaaaleaacqaIYaGmaeaacqaIYaGmaaaakeaa
% cqWGKbazdaWgaaWcbaGaeGOmaidabeaaaaaaaa!52D5!
\[
\frac{{k^2  + d_1^2 }}{{d_1 }} = \frac{{k^2  + d_2^2 }}{{d_2 }}
\]
We bring all of the $k^2$  terms to the left side and 
rearrange as follows:

% MathType!MTEF!2!1!+-
% feaaeaart1ev0aaatCvAUfKttLearuavP1wzZbItLDhis9wBH5garm
% Wu51MyVXgaruWqVvNCPvMCG4uz3bWemv3yPrwynfgDOLeDHXwAJbqe
% gWuDJLgzHbIqYL2zOrhinfgDObss0fgBPngarqqtubsr4rNCHbGeaG
% qiVCI8FfYJH8sipiYdHaVhbbf9v8qqaqFr0xc9pk0xbba9q8WqFfea
% Y-biLkVcLq-JHqpepeea0-as0Fb9pgeaYRXxe9vr0-vr0-vqpWqaae
% aabiGaciaacaqabeaadaabauaaaOqaamaalaaabaGaem4AaS2aaWba
% aSqabeaacqaIYaGmaaaakeaacqWGKbazdaWgaaWcbaGaeGymaedabe
% aaaaGccqGHRaWkcqWGKbazdaWgaaWcbaGaeGymaedabeaakiabg2da
% 9maalaaabaGaem4AaS2aaWbaaSqabeaacqaIYaGmaaaakeaacqWGKb
% azdaWgaaWcbaGaeGOmaidabeaaaaGccqGHRaWkcqWGKbazdaWgaaWc
% baGaeGOmaidabeaaaaa!50EF!
\[
\frac{{k^2 }}{{d_1 }} + d_1  = \frac{{k^2 }}{{d_2 }} + d_2 
\]

% MathType!MTEF!2!1!+-
% feaaeaart1ev0aaatCvAUfKttLearuavP1wzZbItLDhis9wBH5garm
% Wu51MyVXgaruWqVvNCPvMCG4uz3bWemv3yPrwynfgDOLeDHXwAJbqe
% gWuDJLgzHbIqYL2zOrhinfgDObss0fgBPngarqqtubsr4rNCHbGeaG
% qiVCI8FfYJH8sipiYdHaVhbbf9v8qqaqFr0xc9pk0xbba9q8WqFfea
% Y-biLkVcLq-JHqpepeea0-as0Fb9pgeaYRXxe9vr0-vr0-vqpWqaae
% aabiGaciaacaqabeaadaabauaaaOqaaiabdUgaRnaaCaaaleqabaGa
% eGOmaidaaOWaaeWaaeaadaWcaaqaaiabigdaXaqaaiabdsgaKnaaBa
% aaleaacqaIXaqmaeqaaaaakiabgkHiTmaalaaabaGaeGymaedabaGa
% emizaq2aaSbaaSqaaiabikdaYaqabaaaaaGccaGLOaGaayzkaaGaey
% ypa0Jaemizaq2aaSbaaSqaaiabikdaYaqabaGccqGHsislcqWGKbaz
% daWgaaWcbaGaeGymaedabeaaaaa!51E6!
\[
k^2 \left( {\frac{1}{{d_1 }} - \frac{1}{{d_2 }}} \right) = d_2  - d_1 
\]

% MathType!MTEF!2!1!+-
% feaaeaart1ev0aaatCvAUfKttLearuavP1wzZbItLDhis9wBH5garm
% Wu51MyVXgaruWqVvNCPvMCG4uz3bWemv3yPrwynfgDOLeDHXwAJbqe
% gWuDJLgzHbIqYL2zOrhinfgDObss0fgBPngarqqtubsr4rNCHbGeaG
% qiVCI8FfYJH8sipiYdHaVhbbf9v8qqaqFr0xc9pk0xbba9q8WqFfea
% Y-biLkVcLq-JHqpepeea0-as0Fb9pgeaYRXxe9vr0-vr0-vqpWqaae
% aabiGaciaacaqabeaadaabauaaaOqaaiabdUgaRnaaCaaaleqabaGa
% eGOmaidaaOWaaeWaaeaadaWcaaqaaiabdsgaKnaaBaaaleaacqaIYa
% GmaeqaaOGaeyOeI0Iaemizaq2aaSbaaSqaaiabigdaXaqabaaakeaa
% cqWGKbazdaWgaaWcbaGaeGOmaidabeaakiaayIW7cqWGKbazdaWgaa
% WcbaGaeGymaedabeaaaaaakiaawIcacaGLPaaacqGH9aqpcqWGKbaz
% daWgaaWcbaGaeGOmaidabeaakiabgkHiTiabdsgaKnaaBaaaleaacq
% aIXaqmaeqaaaaa!5677!
\[
k^2 \left( {\frac{{d_2  - d_1 }}{{d_2 {\kern 1pt} d_1 }}} \right) = d_2  - d_1 
\]

It follows immediately that 

% MathType!MTEF!2!1!+-
% feaaeaart1ev0aaatCvAUfKttLearuavP1wzZbItLDhis9wBH5garm
% Wu51MyVXgaruWqVvNCPvMCG4uz3bWemv3yPrwynfgDOLeDHXwAJbqe
% gWuDJLgzHbIqYL2zOrhinfgDObss0fgBPngarqqtubsr4rNCHbGeaG
% qiVCI8FfYJH8sipiYdHaVhbbf9v8qqaqFr0xc9pk0xbba9q8WqFfea
% Y-biLkVcLq-JHqpepeea0-as0Fb9pgeaYRXxe9vr0-vr0-vqpWqaae
% aabiGaciaacaqabeaadaabauaaaOqaaiabdUgaRnaaCaaaleqabaGa
% eGOmaidaaOGaeyypa0Jaemizaq2aaSbaaSqaaiabigdaXaqabaGcca
% aMi8Uaemizaq2aaSbaaSqaaiabikdaYaqabaaaaa!4924!
\[
k^2  = d_1 {\kern 1pt} d_2 
\]
If we substitute this result into Equation~(\ref{eq:bothT}), we find

% MathType!MTEF!2!1!+-
% feaaeaart1ev0aaatCvAUfKttLearuavP1wzZbItLDhis9wBH5garm
% Wu51MyVXgaruWqVvNCPvMCG4uz3bWemv3yPrwynfgDOLeDHXwAJbqe
% gWuDJLgzHbIqYL2zOrhinfgDObss0fgBPngarqqtubsr4rNCHbGeaG
% qiVCI8FfYJH8sipiYdHaVhbbf9v8qqaqFr0xc9pk0xbba9q8WqFfea
% Y-biLkVcLq-JHqpepeea0-as0Fb9pgeaYRXxe9vr0-vr0-vqpWqaae
% aabiGaciaacaqabeaadaabauaaaOqaaiabdsfaunaaBaaaleaacqaI
% XaqmaeqaaOGaeyypa0JaeGOmaiJaeqiWda3aaOaaaeaadaWcaaqaai
% abdsgaKnaaBaaaleaacqaIXaqmaeqaaOGaaGjcVlabdsgaKnaaBaaa
% leaacqaIYaGmaeqaaOGaey4kaSIaemizaq2aa0baaSqaaiabigdaXa
% qaaiabikdaYaaaaOqaaiabdEgaNjabdsgaKnaaBaaaleaacqaIXaqm
% aeqaaaaaaeqaaOGaeeiiaaIaeeiiaaIaaGzbVlabbccaGiabbggaHj
% abb6gaUjabbsgaKjabbccaGiaaywW7cqqGGaaicqqGGaaicqWGubav
% daWgaaWcbaGaeGOmaidabeaakiabg2da9iabikdaYiabec8aWnaaka
% aabaWaaSaaaeaacqWGKbazdaWgaaWcbaGaeGymaedabeaakiabdsga
% KnaaBaaaleaacqaIYaGmaeqaaOGaey4kaSIaemizaq2aa0baaSqaai
% abikdaYaqaaiabikdaYaaaaOqaaiabdEgaNjabdsgaKnaaBaaaleaa
% cqaIYaGmaeqaaaaaaeqaaaaa!72D9!
\[
T_1  = 2\pi \sqrt {\frac{{d_1 {\kern 1pt} d_2  + d_1^2 }}{{gd_1 }}} {\rm   }\quad {\rm  and }\quad {\rm   }T_2  = 2\pi \sqrt {\frac{{d_1 d_2  + d_2^2 }}{{gd_2 }}} 
\]
Simplifying, we find
% MathType!MTEF!2!1!+-
% feaaeaart1ev0aaatCvAUfKttLearuavP1wzZbItLDhis9wBH5garm
% Wu51MyVXgaruWqVvNCPvMCG4uz3bWemv3yPrwynfgDOLeDHXwAJbqe
% gWuDJLgzHbIqYL2zOrhinfgDObss0fgBPngarqqtubsr4rNCHbGeaG
% qiVCI8FfYJH8sipiYdHaVhbbf9v8qqaqFr0xc9pk0xbba9q8WqFfea
% Y-biLkVcLq-JHqpepeea0-as0Fb9pgeaYRXxe9vr0-vr0-vqpWqaae
% aabiGaciaacaqabeaadaabauaaaOqaaiabdsfaunaaBaaaleaacqaI
% XaqmaeqaaOGaeyypa0Jaemivaq1aaSbaaSqaaiabikdaYaqabaGccq
% GH9aqpcqaIYaGmcqaHapaCdaGcaaqaamaalaaabaGaemizaq2aaSba
% aSqaaiabigdaXaqabaGccqGHRaWkcqWGKbazdaWgaaWcbaGaeGOmai
% dabeaaaOqaaiabdEgaNbaaaSqabaGccqqGGaaiaaa!50AF!
\[
T_1  = T_2  = 2\pi \sqrt {\frac{{d_1  + d_2 }}{g}} {\rm  }
\]
the result we were seeking!

\section*{Apparatus}

Our Kater pendulum consists of a 1 inch wide by � inch thick 
brass bar, as shown in Figure 2.  We have several lengths 
available.

\begin{figure}[!hbt]     %Kater pendulum apparatus, kater1.eps
\vspace{2in}
\special{eps:kater1.eps}
\caption{The vertical line marks the center of mass.
$a$ is the distance from the end of the bar to the edge of the first hole.
\label{fig:kater1}}
\end{figure}


There are two ways of doing the experiment.  You may try either one.
As you will see, it takes some time to take the data for Method 1, but
the analysis is very simple.  By contrast, it does not take long to
take the necessary data using Method 2, but the analysis is much more
complex.  We aren't sure yet which method is more accurate.  The
experiment is new at CSB/SJU, and there are quite possibly systematic
errors that we do not yet understand.

\subsection*{Method 1}

We attach sliding weights (for the moment, large paper clips) to the
bar, and measure the periods $T_1$ and $T_2$ about each hole as 
functions of
the position of the weights, measured by the centimeter scale taped on
the bar.  It will turn out that at some position of the weights, the
periods will be nearly equal. This position is usually someplace
between the two holes.  One must take some preliminary measurements to
find out where that position is.  Then, one takes careful measurements
of period vs. weight position for several centimeters on either side
of the position at which the two periods are about equal.

Thus one has two sets of data:  Period vs. weight position for each
support point.  Do a least-squares fit to each set, and then plot both
sets, with fits, on the same graph.  The two fit lines will intersect
each other on the graph of Period vs. weight position.  That point of
intersection is the point at which the two periods are exactly equal,
and one can then read that value of the period from the graph.

Once this work is done, the analysis is straightforward:  From the
graph, one finds the period $T$  from the graph and then uses Equation~(\ref{eq:kater1}) to find $g$.  The calculation of the uncertainty is also
straightforward.

\subsection*{Method 2}

It is much easier and quicker to attach weights to the bar in such a way that $T_1$ is nearly equal to $T_2$, and measure both periods accurately.  However, the analysis is considerably more involved.  At the cost of a little algebra, one can find an expression for g that depends on 
both $T_1$ and $T_2$.  We begin by squaring Equation~(\ref{eq:bothT}) above to obtain

% MathType!MTEF!2!1!+-
% feaaeaart1ev0aaatCvAUfKttLearuavP1wzZbItLDhis9wBH5garm
% Wu51MyVXgaruWqVvNCPvMCG4uz3bWemv3yPrwynfgDOLeDHXwAJbqe
% gWuDJLgzHbIqYL2zOrhinfgDObss0fgBPngarqqtubsr4rNCHbGeaG
% qiVCI8FfYJH8sipiYdHaVhbbf9v8qqaqFr0xc9pk0xbba9q8WqFfea
% Y-biLkVcLq-JHqpepeea0-as0Fb9pgeaYRXxe9vr0-vr0-vqpWqaae
% aabiGaciaacaqabeaadaabauaaaOqaaiabdsfaunaaDaaaleaacqaI
% XaqmaeaacqaIYaGmaaGccqGH9aqpdaWcaaqaaiabisda0iabec8aWn
% aaCaaaleqabaGaeGOmaidaaaGcbaGaem4zaCgaamaalaaabaGaem4A
% aS2aaWbaaSqabeaacqaIYaGmaaGccqGHRaWkcqWGObaAdaqhaaWcba
% GaeGymaedabaGaeGOmaidaaaGcbaGaemiAaG2aaSbaaSqaaiabigda
% XaqabaaaaOGaeeiiaaIaaGzbVlabbccaGiabbccaGiabbggaHjabb6
% gaUjabbsgaKjabbccaGiaaywW7cqqGGaaicqqGGaaicqWGubavdaqh
% aaWcbaGaeGOmaidabaGaeGOmaidaaOGaeyypa0ZaaSaaaeaacqaI0a
% ancqaHapaCdaahaaWcbeqaaiabikdaYaaaaOqaaiabdEgaNbaadaWc
% aaqaaiabdUgaRnaaCaaaleqabaGaeGOmaidaaOGaey4kaSIaemiAaG
% 2aa0baaSqaaiabikdaYaqaaiabikdaYaaaaOqaaiabdIgaOnaaBaaa
% leaacqaIYaGmaeqaaaaaaaa!70D8!
\[
T_1^2  = \frac{{4\pi ^2 }}{g}\frac{{k^2  + h_1^2 }}{{h_1 }}{\rm  }\quad {\rm   and }\quad {\rm   }T_2^2  = \frac{{4\pi ^2 }}{g}\frac{{k^2  + h_2^2 }}{{h_2 }}
\]
We solve each of these equations for $k^2$ and equate, as 
follows:

% MathType!MTEF!2!1!+-
% feaaeaart1ev0aaatCvAUfKttLearuavP1wzZbItLDhis9wBH5garm
% Wu51MyVXgaruWqVvNCPvMCG4uz3bWemv3yPrwynfgDOLeDHXwAJbqe
% gWuDJLgzHbIqYL2zOrhinfgDObss0fgBPngarqqtubsr4rNCHbGeaG
% qiVCI8FfYJH8sipiYdHaVhbbf9v8qqaqFr0xc9pk0xbba9q8WqFfea
% Y-biLkVcLq-JHqpepeea0-as0Fb9pgeaYRXxe9vr0-vr0-vqpWqaae
% aabiGaciaacaqabeaadaabauaaaOqaamaalaaabaGaemivaq1aa0ba
% aSqaaiabigdaXaqaaiabikdaYaaakiabdEgaNjaayIW7cqWGKbazda
% WgaaWcbaGaeGymaedabeaaaOqaaiabisda0iabec8aWnaaCaaaleqa
% baGaeGOmaidaaaaakiabgkHiTiabdsgaKnaaDaaaleaacqaIXaqmae
% aacqaIYaGmaaGccqGH9aqpcqWGRbWAdaahaaWcbeqaaiabikdaYaaa
% kiabg2da9maalaaabaGaemivaq1aa0baaSqaaiabikdaYaqaaiabik
% daYaaakiabdEgaNjaayIW7cqWGKbazdaWgaaWcbaGaeGOmaidabeaa
% aOqaaiabisda0iabec8aWnaaCaaaleqabaGaeGOmaidaaaaakiabgk
% HiTiabdsgaKnaaDaaaleaacqaIYaGmaeaacqaIYaGmaaaaaa!6587!
\[
\frac{{T_1^2 g{\kern 1pt} d_1 }}{{4\pi ^2 }} - d_1^2  = k^2  = \frac{{T_2^2 g{\kern 1pt} d_2 }}{{4\pi ^2 }} - d_2^2 
\]
We collect all the terms involving g on the left hand side, 
as follows:

% MathType!MTEF!2!1!+-
% feaaeaart1ev0aaatCvAUfKttLearuavP1wzZbItLDhis9wBH5garm
% Wu51MyVXgaruWqVvNCPvMCG4uz3bWemv3yPrwynfgDOLeDHXwAJbqe
% gWuDJLgzHbIqYL2zOrhinfgDObss0fgBPngarqqtubsr4rNCHbGeaG
% qiVCI8FfYJH8sipiYdHaVhbbf9v8qqaqFr0xc9pk0xbba9q8WqFfea
% Y-biLkVcLq-JHqpepeea0-as0Fb9pgeaYRXxe9vr0-vr0-vqpWqaae
% aabiGaciaacaqabeaadaabauaaaOqaamaalaaabaGaem4zaCgabaGa
% eGinaqJaeqiWda3aaWbaaSqabeaacqaIYaGmaaaaaOWaaeWaaeaacq
% WGubavdaqhaaWcbaGaeGymaedabaGaeGOmaidaaOGaemizaq2aaSba
% aSqaaiabigdaXaqabaGccqGHsislcqWGubavdaqhaaWcbaGaeGOmai
% dabaGaeGOmaidaaOGaemizaq2aaSbaaSqaaiabikdaYaqabaaakiaa
% wIcacaGLPaaacqGH9aqpcqWGKbazdaqhaaWcbaGaeGymaedabaGaeG
% OmaidaaOGaeyOeI0Iaemizaq2aa0baaSqaaiabikdaYaqaaiabikda
% Yaaaaaa!5B1D!
\[
\frac{g}{{4\pi ^2 }}\left( {T_1^2 d_1  - T_2^2 d_2 } \right) = d_1^2  - d_2^2 
\]
%
Or, rearranging, we obtain
%
% MathType!MTEF!2!1!+-
% feaaeaart1ev0aaatCvAUfKttLearuavP1wzZbItLDhis9wBH5garm
% Wu51MyVXgaruWqVvNCPvMCG4uz3bqee0evGueE0jxyaibaieYlf9ir
% Veeu0dXdh9vqqj-hEeeu0xXdbba9frFj0-OqFfea0dXdd9vqaq-Jfr
% VkFHe9pgea0dXdar-Jb9hs0dXdbPYxe9vr0-vr0-vqpWqaaeaabiGa
% ciaacaqabeaadaqaaqaaaOqaamaalaaabaGaeGinaqJaeqiWda3aaW
% baaSqabeaacqaIYaGmaaaakeaacqWGNbWzaaGaeyypa0ZaaSaaaeaa
% cqWGubavdaqhaaWcbaGaeGymaedabaGaeGOmaidaaOGaemizaq2aaS
% baaSqaaiabigdaXaqabaGccqGHsislcqWGubavdaqhaaWcbaGaeGOm
% aidabaGaeGOmaidaaOGaemizaq2aaSbaaSqaaiabikdaYaqabaaake
% aacqWGKbazdaqhaaWcbaGaeGymaedabaGaeGOmaidaaOGaeyOeI0Ia
% emizaq2aa0baaSqaaiabikdaYaqaaiabikdaYaaaaaaaaa!4B33!
\begin{equation}%\[
\frac{{4\pi ^2 }}{g} = \frac{{T_1^2 d_1  - T_2^2 d_2 }}{{d_1^2  - d_2^2 }}
\label{eq:method2a}
\end{equation}%\]
This result is perfectly correct.  But it is not in a form that is
particularly useful in this experiment, where the quantity we can
measure directly is the distance between the supports, 
$d_1+d_2$.  For this
reason, we will try to find quantities $A$ and $B$ that satisfy an
equation of the form
%
% MathType!MTEF!2!1!+-
% feaaeaart1ev0aaatCvAUfKttLearuavP1wzZbItLDhis9wBH5garm
% Wu51MyVXgaruWqVvNCPvMCG4uz3bqee0evGueE0jxyaibaieYlf9ir
% Veeu0dXdh9vqqj-hEeeu0xXdbba9frFj0-OqFfea0dXdd9vqaq-Jfr
% VkFHe9pgea0dXdar-Jb9hs0dXdbPYxe9vr0-vr0-vqpWqaaeaabiGa
% ciaacaqabeaadaqaaqaaaOqaamaalaaabaGaeGinaqJaeqiWda3aaW
% baaSqabeaacqaIYaGmaaaakeaacqWGNbWzaaGaeyypa0ZaaSaaaeaa
% cqWGubavdaqhaaWcbaGaeGymaedabaGaeGOmaidaaOGaemizaq2aaS
% baaSqaaiabigdaXaqabaGccqGHsislcqWGubavdaqhaaWcbaGaeGOm
% aidabaGaeGOmaidaaOGaemizaq2aaSbaaSqaaiabikdaYaqabaaake
% aacqWGKbazdaqhaaWcbaGaeGymaedabaGaeGOmaidaaOGaeyOeI0Ia
% emizaq2aa0baaSqaaiabikdaYaqaaiabikdaYaaaaaGccqGH9aqpca
% aMi8+aaSaaaeaacqWGbbqqaeaacqWGKbazdaWgaaWcbaGaeGymaeda
% beaakiabgUcaRiabdsgaKnaaBaaaleaacqaIYaGmaeqaaaaakiabgU
% caRmaalaaabaGaemOqaieabaGaemizaq2aaSbaaSqaaiabigdaXaqa
% baGccqGHsislcqWGKbazdaWgaaWcbaGaeGOmaidabeaaaaaaaa!5C93!
\[
\frac{{4\pi ^2 }}{g} = \frac{{T_1^2 d_1  - T_2^2 d_2 }}{{d_1^2  - d_2^2 }} = {\kern 1pt} \frac{A}{{d_1  + d_2 }} + \frac{B}{{d_1  - d_2 }}
\]
The algebra takes a few lines, and so we will work it out 
in Appendix 1.  The result is
%
% MathType!MTEF!2!1!+-
% feaaeaart1ev0aaatCvAUfKttLearuavP1wzZbItLDhis9wBH5garm
% Wu51MyVXgaruWqVvNCPvMCG4uz3bqee0evGueE0jxyaibaieYlf9ir
% Veeu0dXdh9vqqj-hEeeu0xXdbba9frFj0-OqFfea0dXdd9vqaq-Jfr
% VkFHe9pgea0dXdar-Jb9hs0dXdbPYxe9vr0-vr0-vqpWqaaeaabiGa
% ciaacaqabeaadaqaaqaaaOqaamaalaaabaGaeGymaedabaGaem4zaC
% gaaiabg2da9maalaaabaGaeGymaedabaGaeGinaqJaeqiWda3aaWba
% aSqabeaacqaIYaGmaaaaaOGaaGjcVpaabmaabaWaaSaaaeaacqWGub
% avdaqhaaWcbaGaeGymaedabaGaeGOmaidaaOGaey4kaSIaemivaq1a
% a0baaSqaaiabikdaYaqaaiabikdaYaaaaOqaaiabikdaYmaabmaaba
% Gaemizaq2aaSbaaSqaaiabigdaXaqabaGccqGHRaWkcqWGKbazdaWg
% aaWcbaGaeGOmaidabeaaaOGaayjkaiaawMcaaaaacqGHRaWkdaWcaa
% qaaiabdsfaunaaDaaaleaacqaIXaqmaeaacqaIYaGmaaGccqGHsisl
% cqWGubavdaqhaaWcbaGaeGOmaidabaGaeGOmaidaaaGcbaGaeGOmai
% ZaaeWaaeaacqWGKbazdaWgaaWcbaGaeGymaedabeaakiabgkHiTiab
% dsgaKnaaBaaaleaacqaIYaGmaeqaaaGccaGLOaGaayzkaaaaaaGaay
% jkaiaawMcaaaaa!5CA3!
\begin{equation}%\[
\frac{1}{g} = \frac{1}{{4\pi ^2 }}{\kern 1pt} \left( {\frac{{T_1^2  + T_2^2 }}{{2\left( {d_1  + d_2 } \right)}} + \frac{{T_1^2  - T_2^2 }}{{2\left( {d_1  - d_2 } \right)}}} \right)
\label{eq:method2b}
\end{equation}%\]
Take a moment to look at this equation, and remember that we are
trying to measure $g$ {\em very} accurately.  It turns out that casting
Equation~(\ref{eq:method2a}) in this form allows us to do so.  The first term in the
brackets contains three quantities---two times, $T_1$ and $T_2$, and
the distance $d_1+d_2$---that we can measure very accurately.  The
second term contains two quantities, $d_1$ and $d_2$, that we know
much less accurately; but if $T_1$ and $T_2$ are nearly equal, the
second term will be much smaller than the first, and so the
uncertainties in $d_1$ and $d_2$ will turn out to matter less.  The
result will be a very accurate measurement!

\subsection*{Procedure (both methods)}

Take a few minutes to look over the apparatus.  The pendulum is
suspended from a knife edge mounted rigidly in a bench clamp.  We
measure the period using a photogate detector connected to the Pasco
Science Workshop software.  Your instructor will show you how this
system works.  Play around with it for a few minutes until you are
comfortable with it.

There are a number of places where systematic error (as opposed to
random error) can creep into this experiment.  {\bf In an experiment
designed for high accuracy, it is important to look for and eliminate
as many sources of systematic error as possible.}  Here are some things
to watch out for:


\begin{enumerate}
\item It is important that the knife edge support be accurately level.
Use a level to check.  If the knife edge is not level, adjust the
support until you get the knife edge as nearly level as you can.

\item The pendulum should oscillate back and forth in a plane.  Be
sure it is not wobbling or moving from side to side.  It's not a bad
idea to start it oscillating, and then let it go for a few minutes,
before you measure the period.

\item The plane in which pendulum oscillates should be perpendicular
to the knife edge.

\item The period of a pendulum depends slightly on the amplitude of
the oscillation.  The equations we use, therefore, are strictly valid
only in the limit of very small amplitude.  Consequently, keep the
amplitude as small as possible.  If time permits, see if you can
detect the small increase in period as the amplitude is increased.

\item The experiment is sensitive to vibration.  For example, try
banging on the table, or even blowing gently on the pendulum, while
you measuring the period.  The effect is striking.  Look for ways to
eliminate vibration as much as possible.  For example, place the
computer mouse on a different surface, so that pressing it to start
the experiment will not introduce a source of vibration.
\end{enumerate}

There are at least two other possibilities for systematic error:  the
quartz clock in the Pasco interface box that is used to measure the
period, and the micrometers and calipers that we use to measure
lengths.

It turns out that many of these sources of systematic error have the
effect of lengthening the period.  Thus, if systematic errors are
present, the value of $g$ will tend to be systematically low.
The procedure is as follows:

{\bf For Method 1:}

\begin{enumerate}
\item Make some initial measurements to find the point at which the periods are approximately equal.

\item For several centimeters on either side of this point, make measurements of period vs. clamp position, at about 0.5 cm to 1 cm. intervals, for both supports.  Be sure to mount the clamps symmetrically on the bar.  For each measurement, record the average of at least 6 or 8 periods and the standard deviation, and calculate the standard deviation of the mean.  For later analysis, it can be convenient to record these measurements directly in the Linfit spreadsheet.

\item For each data set, make a graph and do a least-squares fit.  A fit to a straight line will probably be adequate, but in some cases, the data may show enough curvature that one should fit to a quadratic.  The point where the two curves intersect gives the period needed for Equation~(\ref{eq:kater1}).

\item Measure $d_1+d_2$ (the distance between the two supports).  It is important to make this measurement as accurately as possible.  Your instructor will show you how to do these measurements accurately.  Please handle the calipers and micrometers carefully!
\end{enumerate}

{\bf For Method 2:}

\begin{enumerate}
\item Make an initial measurement of the two periods, fairly roughly.

\item Mount two paper clamps on the pendulum.  The idea here is to find, by trial and error, a location that makes the difference in the periods as small as possible.  Be sure to mount the clamps symmetrically on the bar.  A good choice to start is a position between hole 2 and the center of the bar.  Once you have found an optimum position, record it in your lab notebook.  (Note:  It is possible to make the difference too small-the Pasco software cannot resolve times to better than 0.1 ms, and therefore the periods should differ by at least several times this amount.)

\item With the pendulum oscillating about one support, take about 100 measurements of the period.  Record the average period and the standard deviation, and calculate the standard deviation of the mean.

\item Repeat for the second support.

\item Measure the distances $a$ (the distance from the end of the bar to the first hole), $d_1+d_2$ (the distance between the two supports), and $L$ (the length of the pendulum) as accurately as you can.  Your instructor will show you how to do these measurements accurately.  Please handle the calipers and micrometers carefully!  They are sensitive, and moderately expensive.
\end{enumerate}

{\bf Both methods:}

\begin{enumerate}
\item One can set the Pasco software to stop taking data after a set
elapsed time-thus, one can start timing the pendulum, and the program
will stop itself, and list each reading, the number of oscillations,
the mean value, and the standard deviation.  Your instructor will show
you how this feature of the program works.

\item Notice that the Pasco program calculates the average and the
standard deviation, but does not calculate the standard deviation of
the mean-the appropriate uncertainty for the average period!  You will
need to record the number of periods measured, and calculate the
standard deviation of the mean.  It is worth noting that the standard
deviation-the uncertainty in each period-does not change appreciably,
no matter how many measurements one takes.

\item It can be helpful to adjust y axis of the graph in the Pasco
program so that you can see the variation in period graphically while
the data are being recorded.  Click on the y axis to do so.

\item The micrometer and calipers claim an accuracy of 0.001 inch (or
0.02 mm).  One should, of course, regard such claims with a grain of
salt, and look for ways of double-checking!
\end{enumerate}

\subsection*{Data Analysis}
{\bf Method 1}
From a careful analysis of your graphs, find the point at which the
periods are equal.  Then, use Equation~(\ref{eq:kater1}) to calculate $g$ and the
uncertainty in $g$.

{\bf Method 2}

Refer to Figure 2, and see if you can calculate approximate values of
$d_1$ and $d_2$, given your measured values of $L$ and $a$.  These
values should be measured from the center of mass.  However, it turns
out that the two holes do not cause the center of mass to be shifted
from the center of the bar by more than a half a millimeter or so.
It's easy to confirm this point by balancing the pendulum on one side,
on the knife edge support.  Thus, if you calculate $d_1$ and $d_2$
relative to the center of the bar instead of the center of mass,  you
will know $d_1 - d_2$ to an accuracy of about 1 mm.  It turns out that
this low accuracy measurement will not significantly affect the
accuracy with which $g$ is measured!  You can then calculate 1/$g$
using Equation~(\ref{eq:method2b}).

The uncertainty analysis is a bit complicated, so we will go through
it carefully in Appendix 2.  The idea is to find the uncertainty in
1/$g$ from Equation~(\ref{eq:method2b}), and then use that result to find the
uncertainty in $g$.  See the Appendix for details.  Be sure you
understand why the large uncertainty in $d_1 - d_2$ does not affect
the accuracy with which $g$ is measured, as long as the two periods
are nearly equal.

{\bf Both Methods}

It is important to retain as much accuracy as possible in using either
Equation~(\ref{eq:kater1}) or Equation~(\ref{eq:method2b}) to calculate $g$.  If you are using a
calculator, it is worth using storage registers to store intermediate
results, so that no accuracy is lost in reentering numbers.  You can
also use a spreadsheet, or a program like {\em Mathcad} or {\em Mathematica} to do
the calculations.

\section*{Conclusions}

Report your value of $g$, with uncertainty.  Discuss any systematic
uncertainties or other problems with the experiment that you think
need further refinement.

It might also be interesting to compare your value of $g$ here with
the value you found in the free fall experiment.
\newpage
\begin{center}
{\bf APPENDIX 1
Derivation of Equation~(\ref{eq:method2b})}
\end{center}

We want to derive Equation~(\ref{eq:method2b}) above-that is, we want to find 
the quantities $A$ and $B$ that satisfy
% MathType!MTEF!2!1!+-
% feaaeaart1ev0aaatCvAUfKttLearuavP1wzZbItLDhis9wBH5garm
% Wu51MyVXgaruWqVvNCPvMCG4uz3bWemv3yPrwynfgDOLeDHXwAJbqe
% gWuDJLgzHbIqYL2zOrhinfgDObss0fgBPngarqqtubsr4rNCHbGeaG
% qiVCI8FfYJH8sipiYdHaVhbbf9v8qqaqFr0xc9pk0xbba9q8WqFfea
% Y-biLkVcLq-JHqpepeea0-as0Fb9pgeaYRXxe9vr0-vr0-vqpWqaae
% aabiGaciaacaqabeaadaabauaaaOqaamaalaaabaGaemivaq1aa0ba
% aSqaaiabigdaXaqaaiabikdaYaaakiabdsgaKnaaBaaaleaacqaIXa
% qmaeqaaOGaeyOeI0Iaemivaq1aa0baaSqaaiabikdaYaqaaiabikda
% YaaakiabdsgaKnaaBaaaleaacqaIYaGmaeqaaaGcbaGaemizaq2aa0
% baaSqaaiabigdaXaqaaiabikdaYaaakiabgkHiTiabdsgaKnaaDaaa
% leaacqaIYaGmaeaacqaIYaGmaaaaaOGaeyypa0ZaaSaaaeaacqWGbb
% qqaeaacqWGKbazdaWgaaWcbaGaeGymaedabeaakiabgUcaRiabdsga
% KnaaBaaaleaacqaIYaGmaeqaaaaakiabgUcaRmaalaaabaGaemOqai
% eabaGaemizaq2aaSbaaSqaaiabigdaXaqabaGccqGHsislcqWGKbaz
% daWgaaWcbaGaeGOmaidabeaaaaaaaa!632A!
\begin{equation}%\[
\frac{{T_1^2 d_1  - T_2^2 d_2 }}{{d_1^2  - d_2^2 }} = \frac{A}{{d_1  + d_2 }} + \frac{B}{{d_1  - d_2 }}
\label{eq:partialfrac}
\end{equation}%\]
If we convert the right-hand side of this equation to a common denominator, we obtain
% MathType!MTEF!2!1!+-
% feaaeaart1ev0aaatCvAUfKttLearuavP1wzZbItLDhis9wBH5garm
% Wu51MyVXgaruWqVvNCPvMCG4uz3bWemv3yPrwynfgDOLeDHXwAJbqe
% gWuDJLgzHbIqYL2zOrhinfgDObss0fgBPngarqqtubsr4rNCHbGeaG
% qiVCI8FfYJH8sipiYdHaVhbbf9v8qqaqFr0xc9pk0xbba9q8WqFfea
% Y-biLkVcLq-JHqpepeea0-as0Fb9pgeaYRXxe9vr0-vr0-vqpWqaae
% aabiGaciaacaqabeaadaabauaaaOqaamaalaaabaGaemivaq1aa0ba
% aSqaaiabigdaXaqaaiabikdaYaaakiabdsgaKnaaBaaaleaacqaIXa
% qmaeqaaOGaeyOeI0Iaemivaq1aa0baaSqaaiabikdaYaqaaiabikda
% YaaakiabdsgaKnaaBaaaleaacqaIYaGmaeqaaaGcbaGaemizaq2aa0
% baaSqaaiabigdaXaqaaiabikdaYaaakiabgkHiTiabdsgaKnaaDaaa
% leaacqaIYaGmaeaacqaIYaGmaaaaaOGaeyypa0ZaaSaaaeaacqWGbb
% qqdaqadaqaaiabdsgaKnaaBaaaleaacqaIXaqmaeqaaOGaeyOeI0Ia
% emizaq2aaSbaaSqaaiabikdaYaqabaaakiaawIcacaGLPaaacqGHRa
% WkcqWGcbGqdaqadaqaaiabdsgaKnaaBaaaleaacqaIXaqmaeqaaOGa
% ey4kaSIaemizaq2aaSbaaSqaaiabikdaYaqabaaakiaawIcacaGLPa
% aaaeaacqWGKbazdaqhaaWcbaGaeGymaedabaGaeGOmaidaaOGaeyOe
% I0Iaemizaq2aa0baaSqaaiabikdaYaqaaiabikdaYaaaaaaaaa!6DEF!
\[
\frac{{T_1^2 d_1  - T_2^2 d_2 }}{{d_1^2  - d_2^2 }} = \frac{{A\left( {d_1  - d_2 } \right) + B\left( {d_1  + d_2 } \right)}}{{d_1^2  - d_2^2 }}
\]
Now, in the numerator of the right-hand side, collect all of the terms in $d_1$ and $d_1$:
% MathType!MTEF!2!1!+-
% feaaeaart1ev0aaatCvAUfKttLearuavP1wzZbItLDhis9wBH5garm
% Wu51MyVXgaruWqVvNCPvMCG4uz3bWemv3yPrwynfgDOLeDHXwAJbqe
% gWuDJLgzHbIqYL2zOrhinfgDObss0fgBPngarqqtubsr4rNCHbGeaG
% qiVCI8FfYJH8sipiYdHaVhbbf9v8qqaqFr0xc9pk0xbba9q8WqFfea
% Y-biLkVcLq-JHqpepeea0-as0Fb9pgeaYRXxe9vr0-vr0-vqpWqaae
% aabiGaciaacaqabeaadaabauaaaOqaamaalaaabaGaemivaq1aa0ba
% aSqaaiabigdaXaqaaiabikdaYaaakiabdsgaKnaaBaaaleaacqaIXa
% qmaeqaaOGaeyOeI0Iaemivaq1aa0baaSqaaiabikdaYaqaaiabikda
% YaaakiabdsgaKnaaBaaaleaacqaIYaGmaeqaaaGcbaGaemizaq2aa0
% baaSqaaiabigdaXaqaaiabikdaYaaakiabgkHiTiabdsgaKnaaDaaa
% leaacqaIYaGmaeaacqaIYaGmaaaaaOGaeyypa0ZaaSaaaeaadaqada
% qaaiabdgeabjabgUcaRiabdkeacbGaayjkaiaawMcaaiaaykW7cqWG
% KbazdaWgaaWcbaGaeGymaedabeaakiabgUcaRmaabmaabaGaemOqai
% KaeyOeI0IaemyqaeeacaGLOaGaayzkaaGaaGPaVlabdsgaKnaaBaaa
% leaacqaIYaGmaeqaaaGcbaGaemizaq2aa0baaSqaaiabigdaXaqaai
% abikdaYaaakiabgkHiTiabdsgaKnaaDaaaleaacqaIYaGmaeaacqaI
% YaGmaaaaaaaa!6E2D!
\[
\frac{{T_1^2 d_1  - T_2^2 d_2 }}{{d_1^2  - d_2^2 }} = \frac{{\left( {A + B} \right)\,d_1  + \left( {B - A} \right)\,d_2 }}{{d_1^2  - d_2^2 }}
\]
If we now compare the numerators of the left and right sides, we see at once that
% MathType!MTEF!2!1!+-
% feaaeaart1ev0aaatCvAUfKttLearuavP1wzZbItLDhis9wBH5garm
% Wu51MyVXgaruWqVvNCPvMCG4uz3bWemv3yPrwynfgDOLeDHXwAJbqe
% gWuDJLgzHbIqYL2zOrhinfgDObss0fgBPngarqqtubsr4rNCHbGeaG
% qiVCI8FfYJH8sipiYdHaVhbbf9v8qqaqFr0xc9pk0xbba9q8WqFfea
% Y-biLkVcLq-JHqpepeea0-as0Fb9pgeaYRXxe9vr0-vr0-vqpWqaae
% aabiGaciaacaqabeaadaabauaaaOabaeqabaGaemyqaeKaey4kaSIa
% emOqaiKaeyypa0Jaemivaq1aa0baaSqaaiabigdaXaqaaiabikdaYa
% aaaOqaaiabdkeacjabgkHiTiabdgeabjabg2da9iabgkHiTiabdsfa
% unaaDaaaleaacqaIYaGmaeaacqaIYaGmaaaaaaa!4EAA!
\[
\begin{array}{l}
 A + B = T_1^2  \\ 
 B - A =  - T_2^2  \\ 
 \end{array}
\]
Substituting this result into Equation~(\ref{eq:partialfrac}
), we obtain
% MathType!MTEF!2!1!+-
% feaaeaart1ev0aaatCvAUfKttLearuavP1wzZbItLDhis9wBH5garm
% Wu51MyVXgaruWqVvNCPvMCG4uz3bWemv3yPrwynfgDOLeDHXwAJbqe
% gWuDJLgzHbIqYL2zOrhinfgDObss0fgBPngarqqtubsr4rNCHbGeaG
% qiVCI8FfYJH8sipiYdHaVhbbf9v8qqaqFr0xc9pk0xbba9q8WqFfea
% Y-biLkVcLq-JHqpepeea0-as0Fb9pgeaYRXxe9vr0-vr0-vqpWqaae
% aabiGaciaacaqabeaadaabauaaaOqaamaalaaabaGaemivaq1aa0ba
% aSqaaiabigdaXaqaaiabikdaYaaakiabdsgaKnaaBaaaleaacqaIXa
% qmaeqaaOGaeyOeI0Iaemivaq1aa0baaSqaaiabikdaYaqaaiabikda
% YaaakiabdsgaKnaaBaaaleaacqaIYaGmaeqaaaGcbaGaemizaq2aa0
% baaSqaaiabigdaXaqaaiabikdaYaaakiabgkHiTiabdsgaKnaaDaaa
% leaacqaIYaGmaeaacqaIYaGmaaaaaOGaeyypa0ZaaSaaaeaacqWGub
% avdaqhaaWcbaGaeGymaedabaGaeGOmaidaaOGaey4kaSIaemivaq1a
% a0baaSqaaiabikdaYaqaaiabikdaYaaaaOqaaiabikdaYmaabmaaba
% Gaemizaq2aaSbaaSqaaiabigdaXaqabaGccqGHRaWkcqWGKbazdaWg
% aaWcbaGaeGOmaidabeaaaOGaayjkaiaawMcaaaaacqGHRaWkdaWcaa
% qaaiabdsfaunaaDaaaleaacqaIXaqmaeaacqaIYaGmaaGccqGHsisl
% cqWGubavdaqhaaWcbaGaeGOmaidabaGaeGOmaidaaaGcbaGaeGOmai
% ZaaeWaaeaacqWGKbazdaWgaaWcbaGaeGymaedabeaakiabgkHiTiab
% dsgaKnaaBaaaleaacqaIYaGmaeqaaaGccaGLOaGaayzkaaaaaaaa!750D!
\[
\frac{{T_1^2 d_1  - T_2^2 d_2 }}{{d_1^2  - d_2^2 }} = \frac{{T_1^2  + T_2^2 }}{{2\left( {d_1  + d_2 } \right)}} + \frac{{T_1^2  - T_2^2 }}{{2\left( {d_1  - d_2 } \right)}}
\]
This result leads immediately to Equation~(\ref{eq:method2b}) above.
\newpage
\begin{center}
{\bf Appendix 2
Calculation of the uncertainty in 1/$g$}
\end{center}

We begin by writing Equation~(\ref{eq:method2b}) in the form
% MathType!MTEF!2!1!+-
% feaaeaart1ev0aaatCvAUfKttLearuavP1wzZbItLDhis9wBH5garm
% Wu51MyVXgaruWqVvNCPvMCG4uz3bWemv3yPrwynfgDOLeDHXwAJbqe
% gWuDJLgzHbIqYL2zOrhinfgDObss0fgBPngarqqtubsr4rNCHbGeaG
% qiVCI8FfYJH8sipiYdHaVhbbf9v8qqaqFr0xc9pk0xbba9q8WqFfea
% Y-biLkVcLq-JHqpepeea0-as0Fb9pgeaYRXxe9vr0-vr0-vqpWqaae
% aabiGaciaacaqabeaadaabauaaaOqaamaalaaabaGaeGymaedabaGa
% em4zaCgaaiabg2da9maalaaabaGaemiuaaLaey4kaSIaemyuaefaba
% GaeGinaqJaeqiWda3aaWbaaSqabeaacqaIYaGmaaaaaOGaaGjcVdaa
% !4B2F!
\[
\frac{1}{g} = \frac{{P + Q}}{{4\pi ^2 }}{\kern 1pt} 
\]
where
% MathType!MTEF!2!1!+-
% feaaeaart1ev0aaatCvAUfKttLearuavP1wzZbItLDhis9wBH5garm
% Wu51MyVXgaruWqVvNCPvMCG4uz3bWemv3yPrwynfgDOLeDHXwAJbqe
% gWuDJLgzHbIqYL2zOrhinfgDObss0fgBPngarqqtubsr4rNCHbGeaG
% qiVCI8FfYJH8sipiYdHaVhbbf9v8qqaqFr0xc9pk0xbba9q8WqFfea
% Y-biLkVcLq-JHqpepeea0-as0Fb9pgeaYRXxe9vr0-vr0-vqpWqaae
% aabiGaciaacaqabeaadaabauaaaOqaaiabdcfaqjabg2da9maalaaa
% baGaemivaq1aa0baaSqaaiabigdaXaqaaiabikdaYaaakiabgUcaRi
% abdsfaunaaDaaaleaacqaIYaGmaeaacqaIYaGmaaaakeaacqaIYaGm
% daqadaqaaiabdsgaKnaaBaaaleaacqaIXaqmaeqaaOGaey4kaSIaem
% izaq2aaSbaaSqaaiabikdaYaqabaaakiaawIcacaGLPaaaaaGaeeii
% aaIaaGzbVlabbccaGiabbccaGiabbggaHjabb6gaUjabbsgaKjabbc
% caGiabbccaGiaaywW7cqqGGaaicqqGrbqucqqG9aqpdaWcaaqaaiab
% dsfaunaaDaaaleaacqaIXaqmaeaacqaIYaGmaaGccqGHsislcqWGub
% avdaqhaaWcbaGaeGOmaidabaGaeGOmaidaaaGcbaGaeGOmaiZaaeWa
% aeaacqWGKbazdaWgaaWcbaGaeGymaedabeaakiabgkHiTiabdsgaKn
% aaBaaaleaacqaIYaGmaeqaaaGccaGLOaGaayzkaaaaaiabbccaGaaa
% !6FC0!
\[
P = \frac{{T_1^2  + T_2^2 }}{{2\left( {d_1  + d_2 } \right)}}{\rm  }\quad {\rm   and  }\quad {\rm  Q = }\frac{{T_1^2  - T_2^2 }}{{2\left( {d_1  - d_2 } \right)}}{\rm  }
\]
But the quantity $d_1 + d_2$ is measured as a single distance, with a single uncertainty.  Likewise, one can consider $d_1 - d_2$ as a single quantity, known to within a millimeter or so.  (Note that in any case, one cannot treat $d_1$ and $d_2$ as independent-as one increases, the other necessarily decreases.)  Consequently, it is convenient to define
% MathType!MTEF!2!1!+-
% feaaeaart1ev0aaatCvAUfKttLearuavP1wzZbItLDhis9wBH5garm
% Wu51MyVXgaruWqVvNCPvMCG4uz3bWemv3yPrwynfgDOLeDHXwAJbqe
% gWuDJLgzHbIqYL2zOrhinfgDObss0fgBPngarqqtubsr4rNCHbGeaG
% qiVCI8FfYJH8sipiYdHaVhbbf9v8qqaqFr0xc9pk0xbba9q8WqFfea
% Y-biLkVcLq-JHqpepeea0-as0Fb9pgeaYRXxe9vr0-vr0-vqpWqaae
% aabiGaciaacaqabeaadaabauaaaOqaaiabdseaejabggMi6kabdsga
% KnaaBaaaleaacqaIXaqmaeqaaOGaey4kaSIaemizaq2aaSbaaSqaai
% abikdaYaqabaGccqqGGaaicaaMf8UaeeiiaaIaeeiiaaIaeeyyaeMa
% eeOBa4MaeeizaqMaeeiiaaIaeeiiaaIaaGzbVlabbccaGiabdsgaKj
% abggMi6kabdsgaKnaaBaaaleaacqaIXaqmaeqaaOGaeyOeI0Iaemiz
% aq2aaSbaaSqaaiabikdaYaqabaaaaa!5C79!
\[
D \equiv d_1  + d_2 {\rm  }\quad {\rm   and  }\quad {\rm  }d \equiv d_1  - d_2 
\]
Hence, our expressions for $P$ and $Q$ become
% MathType!MTEF!2!1!+-
% feaaeaart1ev0aaatCvAUfKttLearuavP1wzZbItLDhis9wBH5garm
% Wu51MyVXgaruWqVvNCPvMCG4uz3bWemv3yPrwynfgDOLeDHXwAJbqe
% gWuDJLgzHbIqYL2zOrhinfgDObss0fgBPngarqqtubsr4rNCHbGeaG
% qiVCI8FfYJH8sipiYdHaVhbbf9v8qqaqFr0xc9pk0xbba9q8WqFfea
% Y-biLkVcLq-JHqpepeea0-as0Fb9pgeaYRXxe9vr0-vr0-vqpWqaae
% aabiGaciaacaqabeaadaabauaaaOqaaiabdcfaqjabg2da9maalaaa
% baGaemivaq1aa0baaSqaaiabigdaXaqaaiabikdaYaaakiabgUcaRi
% abdsfaunaaDaaaleaacqaIYaGmaeaacqaIYaGmaaaakeaacqaIYaGm
% cqWGebaraaGaeeiiaaIaaGzbVlabbccaGiabbccaGiabbggaHjabb6
% gaUjabbsgaKjabbccaGiaaywW7cqqGGaaicqqGGaaicqqGrbqucqqG
% 9aqpdaWcaaqaaiabdsfaunaaDaaaleaacqaIXaqmaeaacqaIYaGmaa
% GccqGHsislcqWGubavdaqhaaWcbaGaeGOmaidabaGaeGOmaidaaaGc
% baGaeGOmaiJaemizaqgaaiabbccaGaaa!6361!
\[
P = \frac{{T_1^2  + T_2^2 }}{{2D}}{\rm  }\quad {\rm   and }\quad {\rm   Q = }\frac{{T_1^2  - T_2^2 }}{{2d}}{\rm  }
\]
Notice that all the quantities in $P$ are known very accurately; however, $Q$ is known less accurately, both because the numerator is a difference, and because $d$ is known only to within a millimeter or so.  It is for this reason that one adjusts the periods to make $Q$ as small as possible.

The error calculation is more complicated than can easily be done using the methods in Appendix A of the Physics 191 laboratory manual.  It can nevertheless be shown that
% MathType!MTEF!2!1!+-
% feaaeaart1ev0aaatCvAUfKttLearuavP1wzZbItLDhis9wBH5garm
% Wu51MyVXgaruWqVvNCPvMCG4uz3bWemv3yPrwynfgDOLeDHXwAJbqe
% gWuDJLgzHbIqYL2zOrhinfgDObss0fgBPngarqqtubsr4rNCHbGeaG
% qiVCI8FfYJH8sipiYdHaVhbbf9v8qqaqFr0xc9pk0xbba9q8WqFfea
% Y-biLkVcLq-JHqpepeea0-as0Fb9pgeaYRXxe9vr0-vr0-vqpWqaae
% aabiGaciaacaqabeaadaabauaaaOqaaiabes7aKjaaygW7cqWGqbau
% cqGH9aqpcqWGqbaudaGcaaqaamaabmaabaGaeGOmaiZaaSaaaeaacq
% WGubavdaWgaaWcbaGaeGymaedabeaakiaayIW7cqaH0oazcaaMb8Ua
% emivaq1aaSbaaSqaaiabigdaXaqabaaakeaacqWGubavdaqhaaWcba
% GaeGymaedabaGaeGOmaidaaOGaey4kaSIaemivaq1aa0baaSqaaiab
% ikdaYaqaaiabikdaYaaaaaaakiaawIcacaGLPaaadaahaaWcbeqaai
% abikdaYaaakiabgUcaRmaabmaabaGaeGOmaiZaaSaaaeaacqWGubav
% daWgaaWcbaGaeGOmaiJaaGjcVdqabaGccqaH0oazcaaMb8Uaemivaq
% 1aaSbaaSqaaiabikdaYaqabaaakeaacqWGubavdaqhaaWcbaGaeGym
% aedabaGaeGOmaidaaOGaey4kaSIaemivaq1aa0baaSqaaiabikdaYa
% qaaiabikdaYaaaaaaakiaawIcacaGLPaaadaahaaWcbeqaaiabikda
% YaaakiabgUcaRmaabmaabaWaaSaaaeaacqaH0oazcaaMb8Uaemiraq
% eabaGaemiraqeaaaGaayjkaiaawMcaamaaCaaaleqabaGaeGOmaida
% aaqabaaaaa!78BB!
\[
\delta P = P\sqrt {\left( {2\frac{{T_1 {\kern 1pt} \delta T_1 }}{{T_1^2  + T_2^2 }}} \right)^2  + \left( {2\frac{{T_{2{\kern 1pt} } \delta T_2 }}{{T_1^2  + T_2^2 }}} \right)^2  + \left( {\frac{{\delta D}}{D}} \right)^2 } 
\]
and that
% MathType!MTEF!2!1!+-
% feaaeaart1ev0aaatCvAUfKttLearuavP1wzZbItLDhis9wBH5garm
% Wu51MyVXgaruWqVvNCPvMCG4uz3bWemv3yPrwynfgDOLeDHXwAJbqe
% gWuDJLgzHbIqYL2zOrhinfgDObss0fgBPngarqqtubsr4rNCHbGeaG
% qiVCI8FfYJH8sipiYdHaVhbbf9v8qqaqFr0xc9pk0xbba9q8WqFfea
% Y-biLkVcLq-JHqpepeea0-as0Fb9pgeaYRXxe9vr0-vr0-vqpWqaae
% aabiGaciaacaqabeaadaabauaaaOqaaiabes7aKjabdgfarjabg2da
% 9iabdgfarnaakaaabaWaaeWaaeaacqaIYaGmdaWcaaqaaiabdsfaun
% aaBaaaleaacqaIXaqmaeqaaOGaaGjcVlabes7aKjaaygW7cqWGubav
% daWgaaWcbaGaeGymaedabeaaaOqaaiabdsfaunaaDaaaleaacqaIXa
% qmaeaacqaIYaGmaaGccqGHsislcqWGubavdaqhaaWcbaGaeGOmaida
% baGaeGOmaidaaaaaaOGaayjkaiaawMcaamaaCaaaleqabaGaeGOmai
% daaOGaey4kaSYaaeWaaeaacqaIYaGmdaWcaaqaaiabdsfaunaaBaaa
% leaacqaIYaGmcaaMi8oabeaakiabes7aKjaaygW7cqWGubavdaWgaa
% WcbaGaeGOmaidabeaaaOqaaiabdsfaunaaDaaaleaacqaIXaqmaeaa
% cqaIYaGmaaGccqGHsislcqWGubavdaqhaaWcbaGaeGOmaidabaGaeG
% OmaidaaaaaaOGaayjkaiaawMcaamaaCaaaleqabaGaeGOmaidaaOGa
% ey4kaSYaaeWaaeaadaWcaaqaaiabes7aKjaaygW7cqWGKbazaeaacq
% WGKbazaaaacaGLOaGaayzkaaWaaWbaaSqabeaacqaIYaGmaaaabeaa
% aaa!77CB!
\[
\delta Q = Q\sqrt {\left( {2\frac{{T_1 {\kern 1pt} \delta T_1 }}{{T_1^2  - T_2^2 }}} \right)^2  + \left( {2\frac{{T_{2{\kern 1pt} } \delta T_2 }}{{T_1^2  - T_2^2 }}} \right)^2  + \left( {\frac{{\delta d}}{d}} \right)^2 } 
\]
The uncertainty in 1/$g$ is therefore
% MathType!MTEF!2!1!+-
% feaaeaart1ev0aaatCvAUfKttLearuavP1wzZbItLDhis9wBH5garm
% Wu51MyVXgaruWqVvNCPvMCG4uz3bWemv3yPrwynfgDOLeDHXwAJbqe
% gWuDJLgzHbIqYL2zOrhinfgDObss0fgBPngarqqtubsr4rNCHbGeaG
% qiVCI8FfYJH8sipiYdHaVhbbf9v8qqaqFr0xc9pk0xbba9q8WqFfea
% Y-biLkVcLq-JHqpepeea0-as0Fb9pgeaYRXxe9vr0-vr0-vqpWqaae
% aabiGaciaacaqabeaadaabauaaaOqaaiabes7aKjaaykW7daqadaqa
% amaalaaabaGaeGymaedabaGaem4zaCgaaaGaayjkaiaawMcaaiabg2
% da9maalaaabaWaaOaaaeaacqaH0oazcaaMb8Uaemiuaa1aaWbaaSqa
% beaacqaIYaGmaaGccqGHRaWkcqaH0oazcaaMb8Uaemyuae1aaWbaaS
% qabeaacqaIYaGmaaaabeaaaOqaaiabisda0iabec8aWnaaCaaaleqa
% baGaeGOmaidaaaaaaaa!570D!
\[
\delta \,\left( {\frac{1}{g}} \right) = \frac{{\sqrt {\delta P^2  + \delta Q^2 } }}{{4\pi ^2 }}
\]
and it is an easy matter to find the uncertainty in $g$.
