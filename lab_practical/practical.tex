\newexp

The ability to \underline{collect} useful data (and estimate the accuracy of that data) and the ability to
\underline{analyze} data (interpreting the results of computer programs; making proper plots) are
two goals of this course (see page \pageref{section:purpose}).  This lab practical exam will test
only the most easily-tested skills, namely: interacting with and interpreting the results
of computer programs.  In this exam you will be {\em given} a dataset, told the appropriate functional
form, and, working individually, you will:
\begin{itemize}
\item Enter this data into a spreadsheet (including numbers in scientific notation) 
\item Transfer the data to \WAPP.
\begin{itemize}
\item Tell \WAPP about the errors in the data. All the datasets
assume no $x$-error and generally (but not always) there will be a simple formula for $y$-error.
\item  Tell \WAPP the proper functional form.
\item Tell \WAPP to find the parameters of the best possible fit.
\item Print out the resulting fit report.
\end{itemize}
\item Report the scientific validity of the resulting fit.  (Think reduced $\chi^2$.)
\item Report the resulting best-fit parameters (including proper units, sigfigs, and error).
\item Calculate some quantity based on those parameters and, most importantly,
find the error in that calculated quantity. (That is: error propagation using either the high-low method of
page \pageref{par:high.low.game} or the formula based methods of Table~\ref{table1} and
Appendix E.)  You must show these calculations by
self-documenting your spreadsheet.
\item Produce proper (axes labels, title, etc.) hardcopy plots of the data with fitted curve.
In most cases, two hardcopy plots (one with normal scales, and one in which you have selected scales 
to linearize the curve) are required.  The quadratic functional form cannot be linearized
and the linear functional form is linear using normal scales, so in those two cases you need only turn in
one hardcopy plot.
\end{itemize}
You can find practice exam datasets at: \verb+http://www.physics.csbsju.edu/lab/practical+
Your exam will consist of one of these on-line problems (but with different data). Therefore,
if, before lab, you assure you can do all the practice problems, you should be able to
get 100\% on your actual practical exam!

Your work product for this exam includes:
a (properly formatted, see page \pageref{par:quick.report}) quick report card, 
that reports (units, sigfigs, error) the functional parameters, a
calculated quantity, and assessment of scientific validity (5 points),
hardcopy plots (usually two, 2 points),
hardcopy of your self-documented spreadsheet showing the dataset and 
specified calculations  (2 points),
hardcopy of the \WAPP fit report with all these items nicely stapled together
and handed to your instructor (1 point).
Your lab notebook is not needed for this exam.  You may use this lab manual and the textbook
as references during this exam.

{\bf You will need to sign up for a particular time slot (during your normal lab period) to take this lab practical exam.}