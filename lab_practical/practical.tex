% Lab Practical Exam
\newexp

\section*{Introduction}

Learning to collect  data, to estimate the accuracy of that data, and to
analyze data (using computer programs and interpreting the results; making proper graphs) are
among the goals of this course.  This lab practical exam will examine
only the most easily-tested skills, using and interpreting the results
of computer programs.  In this exam you will be {\em given} a dataset; %told the appropriate functionalfor;  then, working individually, you will:

\begin{itemize}
\item Enter this data into a spreadsheet (including numbers in scientific notation) 
\item Transfer the data to your fitting program.  
\begin{itemize}
\item Tell your fitting program about the errors in the data. All the datasets
assume no $x$-error and generally (but not always) there will be a simple formula for $y$-error.
\item  Determine the appropriate functional form and enter it into your fitting program.
\item Tell your fitting program to find the parameters of the best possible fit.
\item Print out the resulting fit report.
\end{itemize}
\item Report and briefly discuss the scientific validity of the resulting fit.  (Think reduced $\chi^2$.)
\item Report the resulting best-fit parameters (including proper units, signficant figures, and error).
\item Calculate some quantity based on those parameters and, most importantly,
find the error in that calculated quantity. (That is: use error propagation based as needed on the formula based methods of Table~\ref{table1} and
Appendix E.)  Include these calculations on your spreadsheet; they should be well-organized and carefully explained.
\item Produce appropriate  printed graphs (axes labels, title, etc.) of the data with your fitted curve.
In most cases, two graphs (one with normal scales, and one in which you have selected scales 
to linearize the curve) are required.  The quadratic functional form cannot be linearized
and the linear functional form is already linear using normal scales, so in those two cases you need only turn in one printed graph.
\end{itemize}
%You can find practice exam datasets at: \verb+http://www.physics.csbsju.edu/lab/practical+
%Your exam will consist of one of these on-line problems (but with different data). Therefore,
%if, before lab, you assure you can do all the practice problems, you should be able to
%get 100\% on your actual practical exam!

Your report for this exam should include:
\begin{itemize}
\item
A summary sheet that %(properly formatted, see page \pageref{par:quick.report}) quick report card, 
reports  fitting function you have chosen, the functional parameters (units, significant figures, error), a
calculated quantity, and assessment of scientific validity (5 points).  You are welcome to do this summary in a word processor or a spreadsheet.  If your summary is handwritten, be sure it is clear, organized, and legible;
\item
printed plots (usually two, 2 points);
\item
a printed copy of your spreadsheet showing the dataset and 
specified calculations  (2 points);
\item
a printed of the fit report with all these items nicely stapled together
and handed to your instructor (1 point).
\end{itemize}
You will not need your lab notebook for this exam, though you are welcome to use it as a reference.  You may also use this lab manual and the textbook
as references.

{\bf You will need to sign up for a particular time slot (during your normal lab period) to take this lab practical exam.}
